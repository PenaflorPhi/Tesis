\section{Conclusiones}
\begin{frame}
	\frametitle{Conclusiones}
	\begin{enumerate}
		\item En este trabajo se desarrollaron herramientas de la teoría de variedades suaves. Se estudiaron las propiedades de las variedades topológicas y las variedades suaves. \pause
		\item Se trasladaron conceptos del cálculo en espacios euclidianos a las variedades suaves. \pause
		\item Se trabajaron conceptos de geometría Riemanniana. \pause
			\begin{itemize}
				\item Se definieron conceptos como las métricas Riemannianas, las conexiones y las curvas geodésicas. \pause
				\item Se demostró la existencia de las métricas en variedades suaves, así como la existencia y unicidad de las conexiones y las geodésicas. \pause
			\end{itemize}
		\item Se definió el espacio hiperbólico de dimensión $2$ y se encontraron las curvas geodésicas.
	\end{enumerate}
\end{frame}



\begin{frame}[allowframebreaks]
\frametitle{Referencias}
\nocite{lee2013introduction}
\nocite{tu2010introduction}
\nocite{spivak1971calculus}
\nocite{duistermaat2004multidimensional}
\nocite{lee2009manifolds}
\nocite{warner2013foundations}
\nocite{greub2012multilinear}
\nocite{hoffman2015linear}
\nocite{rudin1976principles}
\nocite{gopalakrishnan1984commutative}
\nocite{munkres2000topology}
\nocite{roman2007advanced}
\nocite{lee2019introduction}
\nocite{tao_2016}
\nocite{whitney1936differentiable}
\nocite{tu2017differential}
\nocite{delia2011geodesia}
\nocite{do1992riemannian}
\nocite{biezuner2020geometry}

\printbibliography
\end{frame}
