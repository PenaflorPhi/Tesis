\chapter{Corchetes de Lie}\label{Anexo: Corchetes de Lie}
\begin{definition}[Corchete de Lie]
	Si $X$ e $Y$ son campos vectoriales suaves en una variedad $M$. Definimos al campo vectorial $[X,Y]$ como:
	\[
		[X,Y]f = X(Yf) - Y(Xf),
	\]
	donde $f$ es una función suave en $C^{\infty}(M)$. Al campo vectorial $[X,Y]$ le llamamos el \textit{corchete de Lie}.
\end{definition}

\begin{lemma}
	Si $X$ e $Y$ son campos vectoriales suaves. Entonces, el corchete de Lie $[X,Y]$ cumple las siguientes propiedades:
	\begin{itemize}
		\item $[X,Y]$ es un campo vectorial suave en $M$.
		\item El corchete de Lie es un mapa bilineal.
		\item El corchete de Lie es anticonmutativo.
		\item Si $f,g \in C^{\infty}(M)$. Entonces,
		      \[
			      [fX,gY] = fg[X,Y] + fX(g)Y - gY(f)X
		      \]
		\item El corchete de Lie cumple la identidad de Jacobi, esto es: Para cualesquiera campos vectoriales suaves $X,Y$ y $Z$ en $M$ se cumple:
		      \[
			      [X,[Y,Z]] + [Y,[Z,X]] + [Z,[X,Y]] = 0
		      \]
	\end{itemize}
\end{lemma}

\begin{proof}
	\phantom{ }
	\begin{enumerate}
		\item $[X,Y]$ es un campo vectorial suave, esto es una sencilla consecuencia de la segunda condición del lema \ref{Lemma: Más Criterios de Suavidad Para Campos Vectoriales} y de que la suma de campos vectoriales suaves es un campo vectorial suave.
		\item Mostraremos la linealidad de en la primera coordenada, la linealidad en la segunda coordenada se sigue de manera idéntica.

		      Sean $X, Y$ y $Z$ campos vectoriales suaves, y sean $\alpha, \beta$ números reales cualesquiera, por definición tenemos que:
		      \begin{align*}
			      [aX + bY, Z] & = (aX + bY)Z  - Z(aX + bY) \\
			                   & = aXY + bYZ - aZX - bZY    \\
			                   & = a(XY - ZX) + b(YZ - bZY) \\
			                   & = a[X,Z] + b[Y,Z]
		      \end{align*}
		\item La anticonmutatividad del corchete se puede observar fácilmente a partir de la definición:
		      \[
			      [X,Y] = XY - YX = -(YX + XY) = -[Y,X]
		      \]
		\item Por los resultados observados en la subsección \ref{Subsección: Campos Vectoriales Como Derivaciones} sabemos que los campos vectoriales pueden ser entendidos como derivaciones, utilizando esta definición de los campos vectoriales y la linealidad del corchete tendremos que:
		      \begin{align*}
			      [fX,gY] & = fX(gY) - gY(fX)                   \\
			              & = fg(XY) + fX(g)Y - fgY(X) - gY(f)X \\
			              & = fg[X,Y] + fX(g)Y - gY(f)X
		      \end{align*}
		\item Para mostrar la identidad de Jacobi podemos simplemente expandir cada uno de los términos de la suma:
		      \begin{align*}
			      [X,[Y,Z]] & = [X,YZ - ZY]           \\
			                & = [X,YZ] - [X,ZY]       \\
			                & = XYZ - YZX - XZY + ZYX \\
			      [Y,[Z,X]] & = [Y,ZX - XZ]           \\
			                & = [Y,ZX] - [Y,XZ]       \\
			                & = YZX - ZXY - YXZ + XZY \\
			      [Z,[X,Y]] & = [Z,XY- YX]            \\
			                & = [Z,XY] - [Z,YX]       \\
			                & = ZXY - XYZ - ZYX + YXZ \\
		      \end{align*}
		      No es difícil comprobar que al sumar estos términos todos se anularan, por lo cual:
		      \[
			      [X,[Y,Z]] + [Y,[Z,X]] + [Z,[X,Y]] = 0
		      \]
	\end{enumerate}
\end{proof}
