\section{Fibrados y Campos Tensoriales}\label{Sección: Campos Tensoriales}
En esta sección daremos algunas definiciones y resultados análogos a los
vistos a lo largo del capítulo \ref{Capítulo: Cálculo en Variedades}.
Definiremos lo que son los \it{fibrados tensoriales} y los \it{campos
	tensoriales}, estos son generalizaciones de los conceptos de fibrado tangente y
campos vectoriales, y, de modo similar, nos dan una manera con cual podemos
asociar a cada punto de una variedad suave un tensor. Omitiremos las
demostraciones de los resultados en está sección ya que son exactamente iguales
a los casos ya vistos.

\begin{definition}[Fibrado tensorial]
	Sea $M$ una variedad suave, el \it{fibrado de $(k,l)-$tensores en
		$M$} se definirá como la unión disjunta de todos los espacios formados por
	tensores mixtos	de tipo $(k,l)$ definidos en $M$, esto es,
	\[
		\mathfrak{T}^{k}_{l}(TM) = \bigsqcup_{p \in M} \mathfrak{T}^{k}_{l}(T_{p}M)
	\]
\end{definition}

Mencionamos que los tensores de tipo $(k,l)$ so una generalización de los
$k-$tensores covariantes y de los $l-$tensores contravariantes, es por este
motivo que la definición anterior puede ser divida en dos casos diferentes,
obteniendo lo que sería un \it{fibrado de $k-$tensores covariantes} cuando $l=0$
y un \it{fibrado de $l-$tensores contravariantes} cuando $k=0$, más aún, la
definición de fibrado tensorial también engloba las definiciones de fibrado
tangente y de fibrado cotangente, por está razón tendremos las siguientes
igualdades.
\begin{align*}
	\mathfrak{T}^{0}_{0} & = M \times \K \\
	\mathfrak{T}^{0}_{l} & = TM          \\
	\mathfrak{T}^{k}_{0} & = T^{*}M
\end{align*}
Del mismo modo que ocurre con los fibrados tangentes y los fibrados cotangentes,
hay una proyección natural del fibrado tensorial a la variedad, la cual nos
induce induce una topología que determina una estructura suave, haciendo del
fibrado tensorial un fibrado vectorial sobre $M$.

\begin{definition}
	Sea $M$ una variedad suave, un \it{campo tensorial} es una sección del fibrado
	tensorial $(\mathfrak{T}^{k}_{l}(TM), M, \pi)$, esto quiere decir que un campo
	tensorial es un mapa $A: M \to \mathfrak{T}^{k}_{l}(TM)$ tal que $A(p) \in
		\mathfrak{T}^{k}_{l}(TM)$ para cada $p \in M$. Diremos que el campo es un
	\it{campo suave} si el mapa $A$ es suave.
\end{definition}

Denotaremos al conjunto de campos suaves en una variedad por
$\Gamma^{k}_{l}(M)$, este conjunto es un modulo sobre el anillo de funciones
suaves, por lo que cada elemento $A \in \Gamma^{k}_{l}(M)$ puede ser expresado
como una combinación lineal formal de la siguiente forma: Dadas coordenadas
locales $(\phi_1, \ldots, \phi_n)$ en alguna carta, un campo $A$ puede ser
expresado en cualquier punto de la carta como:
\[
	A = \sum_{i}\sum_{j}
	A_{i_1, \ldots i_k}^{j_1 \ldots, j_k}
	\frac{\partial}{\partial \phi_{i_1}} \ot
	\cdots \ot \frac{\partial}{\partial \phi_{i_k}}
	d\phi_{j_1} \otimes \cdots \otimes d_\phi{j_l},
\]
donde $A_{i_1,\ldots,i_k}^{j_1,\ldots,j_k}$ son funciones suaves,
$\frac{\partial}{\partial \phi_i}$ son las coordenadas locales en la carta y
$d\phi_{j}$ los mapas duales a estas coordenadas. No debería ser sorprendente
que cuando $k=0$ y $l=1$ o $k=1$ y $l=0$ estas representaciones coincidirán con las
representaciones que hemos dado para los campos vectoriales y los campos de
covectores.

Para finalizar daremos caracterizaciones de suavidad para el campo tensorial, la
demostración del teorema se pueden seguir del mismo que como se hizo para el
teorema \ref{Teorema: Primer Criterio de Suavidad Para Campos Vectoriales} y el
lema \ref{Lemma: Más Criterios de Suavidad Para Campos Vectoriales}.

\begin{theorem}[Criterios de suavidad para campos tensoriales]
	Sea $M$ una variedad suave y sea $A: M \to \mathfrak{T}^{k}_{l}(TM)$
	un campo tensorial, las siguientes propiedades son equivalentes.
	\begin{enumerate}
		\item $A$ es un campo suave.
		\item En cada carta coordenada, las funciones componentes de $A$ son
		      suave.
		\item Cada punto de la variedad esta contenido en alguna carta en la cual
		      $A$ tiene funciones componentes suaves.
		\item Si $X_1, \ldots, X_k$ son campos vectoriales e $Y_1, \ldots, Y_l$ son
		      campos de covectores, entonces las funciones $A(X_1, \ldots, X_k,
          Y_1,\ldots,Y_l):
			      M \to \K$ definidas por:
		      \[
			      A(X_1,\ldots,X_k,Y_1,\ldots,Y_l)(p)
			      =
			      A_p 
            \l( 
            \l. X_1 \r|_{p}, 
            \ldots, 
            \l. X_k \r|_{p},
            \l. Y_{1} \r|_{p}, 
            \ldots, 
            \l. Y_{l} \r|_{p} 
            \r),
		      \]
          son funciones suaves.
	\end{enumerate}
\end{theorem}
