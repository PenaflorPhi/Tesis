\section{Productos Tensoriales}
\label{Sección: Productos Tensoriales}
Existen diferentes maneras de definir lo que, con los tensores, nosotros
estudiaremos en esta sección dos maneras de definirlos, cada una con sus
ventajas y desventajas; al final de esta sección probaremos que las dos
maneras son, esencialmente, equivalentes.

La primera definición que daremos es quizá la más sencilla, definiremos a los
tensores como mapas multilineales, la ventaja principal que nos da esta
definición es que podemos trabajar con los tensores con herramientas conocidas
del álgebra lineal y el cálculo; sin embargo, esto tiene la desventaja que las
representaciones de funciones multilineales dependen de las bases de los
espacios vectoriales.

\begin{definition}[Producto Tensorial de Mapas]
	Sean $V_1, \ldots, V_n$ y $W_1, \ldots, W_m$ espacios vectoriales sobre un
	campo $\K$, y sean $F \in L(V_1,\ldots,V_n;\K)$ y $G \in
		L(W_1, \ldots, W_m; \K)$. Definimos a la función:
	\[
		F \ot G:
		V_1 \times \cdots \times V_n \times W_1 \times \cdots \times W_m
		\to \K,
	\]
	como:
	\[
		F \ot G (v_1,\ldots,v_n,w_1,\ldots,w_m)
		= F(v_1,\ldots,v_n)G(w_1,\ldots,w_m).
	\]
\end{definition}
De la multilinealidad de $F$ y $G$ se sigue que $F \ot G$ depende
linealmente de cada uno de los vectores $v_1, \ldots, v_n, w_1, \ldots, w_m$,
por lo cual $F \ot G$ es un elemento de $L(V_1,\ldots,V_n,W_1,\ldots,W_m;
	\K)$, a esta función le llamamos el \textit{producto tensorial de $F$
	y $G$}.

Una consecuencia de que $F$ y $G$ sean elementos de $L(V_1,\ldots,V_n;
	\K)$ y $L(W_1,\ldots, W_m;\K)$ es que el producto tensorial es
un mapa bilineal y asociativo.

Mostraremos que el producto tensorial $\ot: L(V_i;\K) \times
	L(W_j; \K) \to \K$ es lineal en la primera coordenada, la
linealidad en la segunda coordenada se sigue exactamente de la misma manera.

Sean $F, G \in L(V_i;\K)$ y $H \in L(W_j;\K)$ y sean $(v_1,
	\ldots, v_n) \in V_1 \times \cdots \times V_n$, $(w_1, \ldots, w_n) \in W_1 \times
	\cdots \times W_m$ y $a \in \K$, para reducir un poco el tamaño de las
cuentas, denotaremos a los vectores $(v_1,\ldots,v_n)$ y $(w_1,\ldots,w_m)$
por $(v_i)$ y $(w_j)$, respectivamente. Por definición de producto tensorial
tenemos:
\begin{align*}
	(aF + G) \ot H ((v_i),(w_j)) & = (aF + G)(v_i) + H(w_j)                  \\
	                             & = \left[ aF(v_i) + G(v_i) \right] H(w_j)  \\
	                             & = aF(v_i)H(w_j) + G(v_i)H(w_j)            \\
	                             & = a(F \ot H)((v_i),(w_j))
	+ (G \ot H) ((v_i),(w_j))                                                \\
	                             & = (a (F \ot H) + (G \ot H)) ((v_i),(w_j))
\end{align*}

La asociatividad del producto tensorial es una consecuencia de la asociatividad
den el campo $\K$. Sean $V_1, \ldots, V_n$, $W_1,\ldots,W_m$, y $U_1,
	\ldots,U_p$ espacios vectoriales sobre un campo $\K$, y sean
$(v_1, \ldots, v_n) \in V_1 \times \cdots \times V_n$, $(w_1, \ldots, w_m)
	\in W_1 \times \cdots \times W_m$, $(u_1, \ldots, u_p) \in U_1 \times \cdots
	\times U_p$. Como se hizo anteriormente, abreviaremos a estos vectores
simplemente por $(v_i)$, $(w_j)$ y $(u_k)$ respectivamente.

Si $F \in L(V_i;\K)$, $G \in L(W_j;\K)$ y $H \in L(U_k;
	\K)$, tendremos que:
\begin{align*}
	(F \ot G) \ot H((v_i),(w_j),(u_k)) & = (F \ot G)((v_i),(w_j))H(u_k)       \\
	                                   & = \left[ F(v_i)G(w_j) \right] H(u_k) \\
	                                   & = F(v_i)\left[ G(w_j)H(u_k) \right]  \\
	                                   & = F(v_i)(G \ot H)((w_j),(u_k))       \\
	                                   & = F \ot (G \ot H)((v_i),(w_j),(u_k))
\end{align*}

La asociatividad del producto tensorial tiene una consecuencia muy interesante,
y es que, si tenemos $n$ funciones multilineales, $F_1, \ldots, F_n$, el
producto tensorial $F_1 \ot \cdots \ot F_n$ estará bien definido. Este hecho
nos permite enunciar el siguiente corolario.

\begin{corollary}
	Sean $V_1, \ldots V_n$ espacios vectoriales y $V_1^{*}, \ldots, V_{n}^{*}$
	sus espacios duales respectivos. Para cada vector $(v_1, \ldots, v_n) \in V_1
		\times \cdots \times V_n$ tenemos que el producto tensorial:
	\[
		\omega_1 \ot \cdots \ot \omega_n (v_1, \ldots, v_n)
		=
		\omega_1(v_1)\cdots \omega_n(v_n),
	\]
	donde cada $\omega_i$ es un covector arbitrario en $\in V_i^{*}$, para
	$1 \leq i \leq n$.
\end{corollary}

Gracias a este corolario es que podemos encontrar, de manera explícita, una
base para el espacio de funciones multilineales, como veremos con el siguiente
teorema.

\begin{theorem}[Base para el espacio de funciones multilineales]
	\label{Teorema: Base para el espacio de funciones multilineales}
	Sean $V_1, \ldots, V_k$ espacios vectoriales finito dimensionales sobre un
	campo $\K$, con dimensiones $n_1, \ldots, n_k$ respectivamente. Para
	cada $1 \leq j \leq k$ sea $\{E_{1}^{j}, \ldots, E_{n_j}^{j} \}$ una base para el
	espacio $V_j$, y sea $\{\epsilon_{j}^{1}, \ldots, \epsilon_{j}^{n_j}\}$ la
	base correspondiente para el espacio dual $V_j^{*}$. El conjunto
	\[
		\mathcal{B} = \{\epsilon_{1}^{i_1} \ot \cdots \ot \epsilon_{k}^{i_k}
		: 1 \leq i_1 \leq n_1, \ldots, 1 \leq i_k \leq n_k\},
	\]
	es una base para el espacio $L(V_1,\ldots,V_k;\K)$.
\end{theorem}

\begin{proof}
	Demostraremos el caso para dos espacios vectoriales, el caso general se
	realiza exactamente de la misma manera.

	Sean $V$ y $W$ espacios vectoriales finito dimensiones sobre un campo
	$\K$, con dimensiones $n$ y $m$ respectivamente, y sean $\{e_1,
		\ldots, e_n\}$ y $\{f_1, \ldots, f_n\}$ bases para $V$ y $W$. Para cada
	$1 \leq i \leq n$ y cada $1 \leq j \leq m$, definiremos las funciones
	multilineales $F_{ij} \in L(V,W;\mathbb{K})$ para las cuales:
	\[
		F_{ij} (v,w) = v_i w_j, \quad (v,w) \in V \times W,
	\]
	donde $v_i$ y $w_j$ son las $i-$ésimas y $j-$ésimas coordenadas de $v$ y $w$
	con respecto a las bases elegidas. Notemos que si $\{\epsilon^{1}, \ldots,
		\epsilon^{n}\}$ y $\{\phi^{1}, \ldots, \phi^{m}\}$ son las bases
	correspondientes  para los espacios duales $V^{*}$ y $W^{*}$, entonces,
	$F_{ij}$ coincide con el producto tensorial:
	\[
		\epsilon_{i} \ot \phi_{j} (v,w) = \epsilon_{i}(v) \phi_{j}(w)
		= u_{i}v_{j}.
	\]
	Mostraremos que el conjunto de funciones $F_{ij}$ es linealmente
	independiente, para esto tomemos $nm$ elementos del campo, digamos
	$\lambda_{1,1}, \ldots, \lambda_{n,m}$ y supongamos que $F_{ij} \in
		\ker\left(L(V,W;K)\right)$, esto quiere decir que para cada $(v,w) \in
		V \times W$ se tiene:
	\[
		\sum_{i=1}^{n}\sum_{j=1}^{m} \lambda_{i,j} F_{ij} = 0.
	\]
	En particular esto se cumple para los vectores de la base, por lo que:
	\[
		\sum_{i=1}^{n}\sum_{j=1}^{m}\lambda_{i,j}F_{i,j}(e_{\hat{\imath}}, f_{\hat{\jmath}})
		= \lambda_{\hat{\imath},\hat{\jmath}} = 0
	\]
	para cada $1 \leq \hat{\imath} \leq n$, $1 \leq \hat{\jmath} \leq m$, por lo cual, el
	conjunto es linealmente independiente.

	Ahora mostraremos que cada función multilineal $F \in L(V,W;\K)$
	se puede escribir como una combinación lineal de funciones $F_{i,j}$. Sea
	$F \in L(V,W;\K)$, para cada vector $(v,w) \in \K$ tenemos:
	\begin{align*}
		F(u,v) & = F\left(\sum_{i=1}^n v_i e_i,\sum_{j=1}^m w_j f_j \right) \\
		       & = F(v_1 e_1, w_1 f_1) + \cdots + F(v_n e_n, w_m f_m)       \\
		       & = F(e_1,f_1)v_1w_1 + \cdots + F(e_n,f_m) v_n w_m           \\
		       & = F(e_1,f_1)F_{1,1} + \cdots + F(e_n,f_m) F_{n,m}          \\
		       & = \sum_{i=1}^{n} \sum_{j=1}^{m} F_{i,j} F(e_i,f_j).
	\end{align*}
	Por lo tanto, podemos concluir que el conjunto de funciones $F_{ij}$ es
	una base para $L(V,W;\K)$, y, como ya hemos mencionado, estas
	funciones coinciden con el producto tensorial de los elementos de la base
	de los espacios duales.
\end{proof}

Ahora estudiaremos una segunda manera de definir a los tensores, esto será,
como elementos de un producto tensorial de espacios vectoriales. Como veremos
a continuación, esto tiene la ventaja de que los tensores no dependen de las
bases de los espacios vectoriales, por lo cual, este es un acercamiento libre
de coordenadas, sin embargo, esta manera de entender a los vectores también es
más abstracta.

Para poder hablar del producto tensorial de espacios vectoriales es necesario
que hablemos de la propiedad universal, a través de la cual definiremos al
producto tensorial.

\begin{definition}[La propiedad universal]
	Sean $V$ y $W$ espacios vectoriales y sea $\ot: V \times W \to T$, donde
	$T$ es algún espacio vectorial. Diremos que el mapa $\ot$ posee la
	\it{propiedad universal} si satisface las siguientes condiciones:
	\begin{enumerate}
		\item Los vectores $v \ot w$ generan a $T$, donde $v \in V$ y
		      $w \in W$.
		\item Si $\phi$ es algún mapa bilineal de $V \times W$ a cualquier espacio
		      vectorial $P$, entonces existe un mapa lineal único $f: T \to P$ que
		      hace que el diagrama \ref{Diagrama de la propiedad universal} sea
		      conmutativo.
	\end{enumerate}
\end{definition}

\begin{figure}[h]
	\center
	\tikzexternaldisable % Desactiva el precompilado de figuras, ¡No quitar!
\begin{tikzcd}
	V   \times W \arrow[d, "\otimes"'] \arrow[r, "\varphi"] & P \\[16pt]
	T \arrow[ru, "f"']                                        &
\end{tikzcd}
\tikzexternalenable % Restaura la el precompilado de figuras.

	\caption{Diagrama: Propiedad universal.}
	\label{Diagrama de la propiedad universal}
\end{figure}

Esta propiedad es universal en el sentido de que, dado dos espacios vectoriales
siempre existirá un tercer espacio vectorial y un mapa bilineal, único hasta
isomorfismo, que vaya del producto cartesiano de los dos espacios al tercer
espacio vectorial, el cual posea la propiedad universal. Nos gustaría esto, es
con este fin que demostraremos los siguientes lemas.

\begin{lemma}
	Sean $V, W$ y $T$ espacios vectoriales, $\ot: V \times W \to T$ un mapa
	bilineal que posee la propiedad universal, y sean $\{v_1,\ldots,v_n\}$ y
	$\{w_1,\ldots,w_n\}$ vectores linealmente independientes en $V$ y $W$
	respectivamente. La relación:
	\[
		\sum_{i=1}^{n} v_i \ot w_i = 0
	\]
	implica que $v_i = 0$ o que $w_i = 0$ para cada $1 \leq i \leq n$.
\end{lemma}

\begin{proof}
	Mostraremos que, si la relación se cumple, y los vectores $v_i$ no son
	necesariamente nulos, entonces cada $w_i$ debe ser nulo.

	Dado que los vectores $\{v_1,\ldots, v_n\}$ son linealmente independientes,
	podemos construir $n$ funciones $f_j: V \to \{0,1\}$ tales que:
	\[
		f_j (v_i) = \delta_{ij}
	\]
	También definiremos una función bilineal $\phi: V \times W \to P$, donde $P$
	es algún espacio vectorial. Definimos a $\phi$ de la siguiente manera:
	\[
		\phi(x,y) = \sum_{j=1}^{n} f_j(x)g_j(y), \quad x \in V, y \in W,
	\]
	donde $g_i$ son mapas lineales arbitrarios, definidos en $W$.

	Dado que $\ot$ posee la propiedad universal por hipótesis, existe un mapa
	lineal $f: T \to P$ tal que para cada $x \in V$ y cada $y \in W$ se tiene:
	\[
		f(x \ot y) = \phi(x,y) = \sum_{j=1}^{n} f_j(x)g_j(y)
	\]
	Si evaluamos el mapa $f$ en $\sum_{i=1}^{n} v_j \otimes w_j$ por la
	linealidad de $f$ y como hemos definido a las funciones $f_j$, se tendrá la
	siguiente cadena de igualdades.
	\begin{align*}
		f\l( \sum_{i=1}^{n} v_j \ot w_j \r)
		 & = \sum_{j=1}^{n} \l( \sum_{i=1}^{n} f_j(v_i) \ot g_j(w_i) \r) \\
		 & = \sum_{j=1}^{n} \sum_{i=1}^{n} \delta_{ij} g_j(w_i)          \\
		 & = \sum_{i=1}^{n} g_i(w_i) = 0
	\end{align*}
	Hemos dicho que las funciones $g_j$ son funciones lineales cualesquiera,
	por lo que podemos concluir que $w_i = 0$ para cada $1 \leq i \leq n$.
\end{proof}

\begin{lemma}
	Sea $\{e_\alpha\}_{\alpha \in A}$ una base para $V$, donde $A$ es algún
	conjunto indicador. Cada vector $z \in T$ puede ser escrito como:
	\[
		z = \sum_{\alpha} e_{\alpha} \ot w_{\alpha}, \quad w_{\alpha} \in W,
	\]
	donde a lo más, una cantidad finita de vectores $w_\alpha$ son no nulos.
	Los vectores $w_\alpha$ están determinados de manera única por el vector $z$.
\end{lemma}

\begin{proof}
	Dado que $\ot$ posee la propiedad universal, los elementos $x \otimes y$
	generan a $T$, lo cual implica que cada $z \in T$ puede ser escrito como la
	suma finita:
	\[
		z = \sum_{i=1}^{n} x_i \ot y_i, x_i \in V, y_i \in W.
	\]
	Expresaremos a cada $x_i$ como una combinación lineal de sus vectores bases,
	\[
		x_i = \sum_{\alpha}\lambda_{\alpha}^{i} e_\alpha,
		\quad
		\lambda_{\alpha}^{i} \in \K,
	\]
	de esta manera podemos escribir al vector $z$ como:
	\[
		z = \sum_{i=1}^{n}
		\l( \sum_{\alpha}\lambda_{\alpha}^{i} e_{\alpha} \r) \ot y_i,
	\]
	y dado que el mapa $\ot$ es bilineal la siguiente cadena de igualdades es
	verdadera:
	\begin{align*}
		z & = \sum_{\alpha}\sum_{i=1}^{n} \lambda_{\alpha}^{i} e_\alpha \ot y_i \\
		  & = \sum_{\alpha}\sum_{i=1}^{n} e_\alpha \ot \lambda_{\alpha}^{i} y_i \\
		  & = \sum_{\alpha} e_\alpha \ot \sum_{i=1}^{n}\lambda_{\alpha}^{i} y_i \\
		  & = \sum_{\alpha} e_\alpha \ot w_\alpha,
	\end{align*}
	esta última igualdad se obtiene simplemente de realizar la sustitución
	$w_\alpha = \sum_{i=1}^{n} \lambda_{\alpha}^{i} y_i$.

	Para probar la unicidad de los vectores $w_\alpha$ supongamos que existen
	vectores $\{w'_{\alpha}\}_{\alpha \in A}$ tales que:
	\[
		z = \sum_{\alpha} e_{\alpha} \ot w_\alpha
		= \sum_{\alpha} e_{\alpha} \ot w'_\alpha.
	\]
	Naturalmente esto implica que:
	\[
		\sum_{\alpha} e_{\alpha} \ot w_\alpha
		- \sum_{\alpha} e_{\alpha} \ot w'_\alpha = 0
	\]
	Y por la linealidad en la segunda coordenada del mapa $\ot$ obtenemos:
	\[
		\sum_{\alpha} e_{\alpha} \ot (w_{\alpha} - w'_{\alpha}) = 0
	\]
	Por el lema anterior podemos concluir que $w_{\alpha} = w'_{\alpha}$
	para cada $\alpha \in A$.
\end{proof}

\begin{lemma}
	Todo vector $z \in T$ puede ser escrito de la forma:
	\[
		z = \sum_{i=1}^{n} v_i \ot w_i, \quad v_i \in V, w_i \in W,
	\]
	donde $\{v_1,\ldots,v_n\}$ y $\{w_1,\ldots,w_n\}$ son vectores linealmente
	independientes.
\end{lemma}

\begin{proof}
	Sabemos que podemos representar a cada vector $z \in T$ como una suma finita
	\[
		z = \sum_{i=1}^{n} v_i \ot w_i.
	\]
	Nuestra intuición, al tener el vector una forma muy similar a la de una
	combinación lineal, debería decirnos que si $\{v_1, \ldots, v_n\}$a y
	$\{w_1, \ldots, w_n\}$ son linealmente independientes entonces $n$ debería
	ser el mínimo entero positivo para el cual es posible escribir a $z$ como
	una suma de esta forma.

	Elijamos pues, vectores $\{v_1,\ldots,v_n\}$ y $\{w_1,\ldots,w_n\}$ tales que
	$n$ sea minimizado.

	Si $n=1$, la independencia lineal se cumple trivialmente. Si por otra parte
	ocurre que $n \geq 2$ podemos proceder por contradicción. Supongamos que el
	conjunto $\{v_1, \ldots, v_n\}$ es linealmente independiente, si este es el
	caso, podemos suponer que:
	\[
		v_n = \sum_{i=1}^{n-1} \lambda_i v_i, \quad \lambda_{i} \in \K
	\]
	Por lo tanto, podremos escribir al vector $z$ como:
	\begin{align*}
		z & = \sum_{i=1}^{n} v_i \ot w_i                                        \\
		  & = \sum_{i=1}^{n-1} v_i \ot w_i + v_n \ot w_n                        \\
		  & = \sum_{i=1}^{n-1}v_i\ot w_i +\sum_{i=1}^{n-1}\lambda_i v_i \ot w_n \\
		  & = \sum_{i=1}^{n-1}v_i\ot (w_i + \lambda_i w_n)
		= \sum_{i=1}^{n-1} v_i \ot w'_i
	\end{align*}
	Pero esto querría decir que los vectores que hemos elegidos no minimizan a
	$n$, lo cual es una contradicción. Por lo tanto, los vectores $\{v_1,
		\ldots, v_n\}$ deben ser linealmente independientes.
	De manera análoga se demuestra que los vectores $\{w_1,\ldots, w_n\}$ son
	linealmente independientes.
\end{proof}

\begin{definition}[Producto tensorial de dos espacios vectoriales]
	Sean $V$ y $W$ espacios vectoriales. El \it{producto tensorial} de $V$ y $W$
	es un par $(T, \ot)$ donde $T$ es un espacio vectorial y $\ot: V \times W \to
		T$ es un mapa bilineal que posee la propiedad universal.
\end{definition}

\begin{theorem}
	Dado dos espacios vectoriales, su producto tensorial existe y es único hasta
	isomorfismo.
\end{theorem}

\begin{proof}
	Comenzaremos mostrando la unicidad. Sean $V$ y $W$ espacios vectoriales,
	supongamos que $(T,\ot)$ y $(T',\op)$ son productos tensoriales. Dado que
	tanto $\ot$ como $\op$ poseen la propiedad universal los siguientes diagramas
	serán conmutativos.
	\begin{figure}[h]
		\center
		\tikzexternaldisable % Desactiva el precompilado de figuras, ¡No quitar!
\begin{tikzcd}
	V \times W \arrow[r, "\oplus"] \arrow[d, "\otimes"'] & T' & V \times W
	\arrow[d, "\oplus"'] \arrow[r, "\otimes"] & T \\[12pt]
	T \arrow[ru, "f"']                                   &    & T' \arrow[ru, "g"']                                  &
\end{tikzcd}
\tikzexternalenable % Restaura la el precompilado de figuras.

	\end{figure}
	Por definición de la propiedad universal sabemos que tanto $f$ como $g$ son
	únicos hasta isomorfismo, así tendremos que $f\ot = \op$ y $g\op = \ot$. De
	aquí que $gf(\ot) = \ot = \id_{T}(\ot)$ y $fg(\op) = \op = \id_{T'}(\op)$,
	con lo cual podemos concluir que $gf =\id_{T}$ $fg = \id_{T'}$, por lo tanto
	$T \cong T'$.

	Ahora mostraremos la existencia. Consideremos el espacio generado por todas
	las combinaciones lineales formales de $V \times W$, denotaremos a este
	conjunto por $F(V \times W)$. Definiremos un subespacio de $F(V \times W)$,
	al cual denotaremos por $G(V \times W)$, este será el subespacio generado por
	todos los elementos de la forma:
	\begin{align*}
		(\lambda v_1 + \mu v_2, w) & - \lambda(v_1, w) - \mu(v_2, w) \\
		(v, \lambda w_1 + \mu w_2) & - \lambda(v, w_1) - \mu(v, w_2) \\
	\end{align*}
	Sean $T$ el espacio cociente $F(V \times W) / G(V \times W)$ y $\pi:
		F(V \times W) \to T$ el mapa proyección. Definiremos el mapa $\ot: V
		\times W \to T$ como:
	\[ v \times w = \pi(v,w) \]
	Mostraremos que $\ot$ posee la propiedad universal. Claramente el mapa $\ot$
	es bilineal, para mostrar que los vectores $v \ot w$ generan a $T$ observemos
	que, por los resultados mostrados anteriormente, cada vector $z \in T$ puede
	ser escrito como una suma finita:
	\[ z = \pi \l(\sum_{i=1}^n\sum_{j=1}^{n} \lambda_{ij}(v_i,w_j)\r) \]
	Por lo tanto, se sigue que:
	\[
		z = (\sum_{i=1}^n\sum_{j=1}^{m} \lambda_{ij}\pi(v_i,w_j)
		= \sum_{i=1}^n\sum_{j=1}^{m} \lambda_{ij} v_i \times w_j,
	\]
	por lo cual, $T$ es generado por los vectores $v \ot w$.

	Ahora, para mostrar la segunda condición consideremos un espacio vectorial
	$P$ y un mapa bilineal $\phi: V \times W \to P$. Podemos extender $\phi$ a
	un mapa lineal $g: F(V \times W) \to P$ definiendo:
	\[
		g\l(\sum_{i=1}^{n}\sum_{j=1}^{m}\lambda_{ij} (v_i,w_j)\r)
		=
		\sum_{i=1}^{n}\sum_{j=1}^{m} \lambda_{ij}f(v_i,w_j)
	\]
	Notemos que para cada $z \in G(V \times W)$ se tendrá que $g(z) = 0$, esto
	dado que si $z$ es un elemento generador de $G(V \times W)$, entonces:
	\begin{align*}
		g(z) & = g((\lambda v_1 + \mu v_2, w) - \lambda(v_1,w)   -\mu(v_2,w))    \\
		     & = g(\lambda v_1 + \mu v_2,  w) - \lambda g(v_1,w) -\mu g(v_2,w)   \\
		     & = \phi(\lambda v_1+\mu v_2, w) -\lambda\phi(v_1,w)-\mu\phi(v_2,w) \\
		     & = 0
	\end{align*}
	Se concluye que $G(V \times W) \subset \ker(g)$, por lo cual $g$ debe inducir un mapa
	lineal
	\[ f: F(V \times W) / G (V \times W) \to P, \]
	tal que:
	\[
		f \circ \pi = g.
	\]
	Esto muestra que:
	\[
		(f \circ \ot)(v,w) = g(v,w) = \phi(v,w)
	\]
	Así, $\otimes$ posee la propiedad universal, por lo cual $(T, \ot)$ es el
	producto tensorial de $V$ y $W$.
\end{proof}

La idea del producto tensorial de espacios vectoriales puede ser extendida de
manera muy natural al producto de $n$ espacios vectoriales, al igual que ocurre
en el caso de dos espacios, este producto tensorial siempre existe y es único
hasta isomorfismo, la demostración se realiza de la misma manera.

\begin{definition}[Propiedad universal]
	Sean $V_1, \ldots, V_n$ espacios vectoriales y $\ot: V_1 \times \cdots \times
		V_n \to T$ un mapa multilineal. Diremos que el mapa $\ot$ posee la propiedad
	universal si:
	\begin{itemize}
		\item Los vectores $v_1 \ot\cdots\ot v_n$ genera a $T$, donde $v_i\in V_i$.
		\item Cada mapa multilineal $\phi: V_1 \times \cdots \times V_n \to P$,
		      donde $P$ es un espacio vectorial, puede ser escrito de la forma:
		      \[ \phi(v_1,\ldots,v_n) = f(v_1 \otimes \cdots \otimes v_n), \]
		      donde $f: T \to P$ es un mapa lineal.
	\end{itemize}
\end{definition}

\begin{definition}[Producto tensorial de espacios vectoriales]
	Sean $V_1, \ldots, V_n$ espacios vectoriales, el \it{producto tensorial} de
	$V_1,\ldots,V_n$ es un par $(T, \ot)$, donde $\ot: V_1 \times \cdots \times
		V_n \to T$ es un mapa multilineal que posee la propiedad universal. Usualmente
	nos referimos simplemente a $T$ como el producto tensorial de los espacios
	vectoriales y denotamos a $T$ como $V_1 \ot \cdots \ot V_n$.
\end{definition}

\begin{theorem}[Base para el producto tensorial de espacios vectoriales]
	\label{Teorema: Base para el producto tensorial}
	Sean $V_1, \ldots, V_k$ espacios vectoriales finito dimensionales, con
	dimensiones $n_1, \ldots, n_k$, respectivamente. Para cada $1 \leq j \leq k$,
	supongamos que $\{E_{1}^{j}, \ldots E_{n_j}^{j}\}$ es una base para $V_j$.
	Entonces, una base para el producto tensorial $V_1 \ot \cdots \ot V_k$ es
	el conjunto:
	\[
		\mathcal{C} = \{ E_{i_1}^{1} \ot \cdots \ot E_{i_k}^{k} :
		1 \leq i_1 \leq n_1, \ldots, 1 \leq i_n \leq n_k \}
	\]
\end{theorem}

\begin{proof}
	Mostraremos el caso para el producto tensorial de dos espacios vectoriales.
	Sean $V$ y $W$ espacios vectoriales, $\{v_1,\ldots,v_n\}$ y $\{w_1,\ldots,
		w_n\}$ bases para $V$ y $W$ respectivamente, como hemos visto, cada vector
	$(v \otimes w) \in V \otimes W$ puede ser expresado como:
	\[
		v \ot w = \sum_{i=1}^{n}\sum_{j=1}^{m} \lambda_i \mu_j v_i \ot w_j,
		\quad \lambda_i, \mu_j \in \K
	\]
	donde los escalares $\lambda_i$ y $\mu_j$ están determinados de forma única
	por la representación de $v$ y $w$ como combinaciones lineales:
	\[
		v = \sum_{i=1}^{n} \lambda_i v_i, \quad w = \sum_{j=1}^{n} \mu_i w_i
	\]

	Ahora mostraremos que el conjunto de vectores $\{v_i \otimes w_j\}$ es
	linealmente independiente. Procederemos de modo similar a como se hizo en la
	demostración del teorema \ref{Teorema: Base para el espacio de funciones
		multilineales}.

	Para cada $1 \leq i \leq n$ y cada $1 \leq j \leq m$ definiremos $n m$
	bilineales $\Phi_{i j}: V \times W \to \K$, estas funciones estarán dadas por:
	\[
		\Phi_{ij}(x,y) = x_i y_j, \quad x \in V, y \in W,
	\]
	donde $x_i$ e $y_j$ son las $i-$ésima y $j-$ésimas coordenadas de $x$ e $y$
	con respecto a las bases elegidas. Dado que cada uno de los mapas $\Phi_{ij}$ es
	bilineal y que el mapa $\ot$ posee la propiedad universal, existirán $nm$
	funciones lineales tales que:
	\[
		\phi_{ij}(v \ot w) = \Phi_{ij}(v,w)
	\]
	Al evaluar cada una de estas funciones $\phi_{ij}$ en cualquier vector del
	kernel, digamos:
	\[
		\sum_{i=1}^{n}\sum_{j=1}^{m} \lambda_i \mu_j (v_i \otimes w_j) = 0,
	\]
	obtendremos que:
	\begin{align*}
		\phi_{\hat{\imath}\hat{\jmath}}\l(\sum_{i=1}^{n}\sum_{j=1}^{m}\lambda_i\mu_j(v_i \ot w_j)\r)
		 & = \sum_{i=1}^{n}\sum_{j=1}^{m}\lambda_i\mu_j\phi_{\hat{\imath}\hat{\jmath}}(v_i\ot w_j) \\
		 & = \sum_{i=1}^{n}\sum_{j=1}^{m}\lambda_i\mu_j\Phi_{\hat{\imath}\hat{\jmath}}(v_i,w_j)
		= \lambda_{\hat{\imath}} \mu_{\hat{\jmath}} = 0
	\end{align*}
	Por lo tanto, podemos concluir que el conjunto de vectores $v_i \ot w_j$ es
	linealmente independiente, y por lo tanto forman una base para $V \ot W$.
\end{proof}

Para terminar con esta sección probaremos la afirmación que hicimos al
principio, que ambas maneras de definir a los tensores, tanto como funciones
multilineales como elementos de productos tensoriales son esencialmente
equivalentes.

\begin{theorem}
	\label{Teorema: Isomorfismo Entre Tensores y Funciones}
	Si $V_1, \ldots, V_n$ son espacios vectoriales sobre un campo $\K$, entonces,
	existen isomorfismos canónicos:
	\[ V_1^* \ot \cdots \ot V_n^* \cong L(V_1, \ldots, V_n; K) \]
	y
	\[ V_1 \ot \cdots \ot V_n \cong L(V_1^*, \ldots, V_n^*; K) \]
\end{theorem}

\begin{proof}
	Realizaremos la demostración para el primer isomorfismo, el segundo
	isomorfismo se sigue de manera similar.

	Sea $\phi: V_1^{*} \times \cdots \times V_n^{*} \to L(V_1, \ldots ,V_n; \K)$
	la función multilineal definida por la relación:
	\[
		\phi(\omega_1,\ldots,\omega_n)(v_1,\ldots,v_n)
		=
		\omega(v_1)\cdots\omega(v_n)
	\]
	Por la propiedad universal de los productos tensoriales sabemos que debe
	existir una función lineal única $f: V_1 \ot \cdots \ot V_n \to L(V_1, \ldots
		V_n; K)$ para el cual, el diagrama \ref{Diagrama: Isomorfismo entre el
		producto tensorial y el espacio de funciones multilineales} es conmutativo.
	\begin{figure}[h]
		\center
		\tikzexternaldisable % Desactiva el precompilado de figuras, ¡No quitar!
\begin{tikzcd}
	V_{1}^{*} \times \cdots \times V_{n}^{*} \arrow[rd, "\otimes"', shift right]
	\arrow[rr, "\varphi"] &
	& {L(V_1,\ldots ,V_n; \mathbb{K})} \\[24pt]
	& V_{1}^{*} \otimes \cdots \otimes V_{n}^{*} \arrow[ru, "f"'] &
\end{tikzcd}
\tikzexternalenable % Restaura la el precompilado de figuras.

		\caption{Diagrama: Isomorfismo entre el producto tensorial y el espacio de
			funciones multilineales.}
		\label{Diagrama: Isomorfismo entre el
			producto tensorial y el espacio de funciones multilineales}
	\end{figure}
	Esto quiere decir que para cada $(v_1, \ldots, v_n) \in V_1 \times \cdots
		\times V_n$ tendremos:
	\[
		f(\omega_1 \ot \cdots \ot \omega_n)(v_1,\ldots,v_n)
		=
		\omega_1(v_1) \cdots \omega_n(v_n)
	\]
	Esta función será un isomorfismo entre los espacios vectoriales $V_1 \ot
		\cdots \ot V_n$ y $L(V_1,\ldots,V_n; \K)$, dado que es lineal, inyectiva para
	cada vector $(v_1, \ldots, v_n) \in V_1 \times \cdots \times V_n$ y que, como
	vimos con los teoremas \ref{Teorema: Base para el espacio de funciones
		multilineales} y \ref{Teorema: Base para el producto tensorial}, ambos
	espacios tienen la misma dimensión, a saber, su dimensión es el producto de
	cada una de las dimensiones de los espacios vectoriales $V_1, \ldots, V_n$.
\end{proof}

Gracias a este teorema podemos referirnos de manera indistinta a los espacios
$L(V_1,\ldots,V_n;K)$ y $V_1^* \ot \cdots \ot V_n^*$ de manera indistinta, y
únicamente haremos la distinción si es que es necesario.
