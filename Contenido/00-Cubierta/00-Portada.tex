% Por alguna razón las guías son lo que mantienen el documento en su lugar. Para mosrtar o desaparecer las guías es necesario comentar el primer input. 
% Guías-0 son las guías completamente transparentes.

\thispagestyle{empty} % Esto deja la página sin numeración.
\tikzexternaldisable % Desactiva el precompilado de figuras, ¡No quitar!
\newgeometry{left=0.7cm, right=0.7cm, top=1.2cm, bottom=0.7cm} % Márgenes únicamente para esta sección.

% Logosímbolo institucional.
\begin{figure}
	\includegraphics[height=4.3cm,right=19.375cm]{Figuras/0-UV/Logo.png}
\end{figure}

% Linea bajo el logosímbolo
\begin{tikzpicture}[remember picture,overlay]
	\draw[opacity=1]($(current page.north west)+(15.125cm,-5.60cm)$)--($(current page.north east)+(-1.5cm,-5.60cm)$);
\end{tikzpicture}

% Datos de la tesis.
\begin{textblock*}{15cm}(2.5cm,5cm)
	\begin{flushright}
		{\GilliusUno \facultad \\ \region}\\[12pt]
		{\GilliusDos \programa}\\[16pt]
		{\GilliusTres \titulo}\\[16pt]
		{\GilliusDos Tesis para acreditar la Experiencia Recepcional}\\[12pt]
		{\GilliusDos Presenta: \\ \textbf{\alumno}}\\[12pt]
		{\GilliusDos Directores: \\ \DirectorUno \\ \DirectorDos}\\[12pt]
		{\GilliusDos \today}\\[12pt]
		{\GilliusDos “Lis de Veracruz: Arte, Ciencia, Luz”}\\[12pt]
	\end{flushright}
\end{textblock*}

% Adorno al pie de página.
\begin{figure}[!b]
	\includegraphics[height=9cm,left]{Figuras/0-UV/Inferior.png}
\end{figure}

% Guías de alineado.
% \begin{tikzpicture}[remember picture,overlay]
	\draw[,dash pattern=on 10pt off 3pt,line width=2pt]($(current page.north)+(6.795cm,0cm)$)--($(current page.south)+(6.795cm,0cm)$);

	\draw[,dash pattern=on 10pt off 3pt,line width=2pt]($(current page.north west)+(0.7cm,0cm)$)--($(current page.south west)+(0.7cm,0cm)$);

	\draw[line width=2pt,purple,latex-latex,,dash pattern=on 10pt off 3pt]($(current page.north west)+(0cm,-1.2cm)$)--($(current page.north east)+(0cm,-1.2cm)$);

	\draw[line width=2pt,purple,latex-latex,,dash pattern=on 10pt off 3pt]($(current page.north west)+(0cm,-5.5cm)$)--($(current page.north east)+(0cm,-5.5cm)$);

	\draw[line width=2pt,purple,latex-latex,,dash pattern=on 10pt off 3pt]($(current page.north west)+(0cm,-6cm)$)--($(current page.north east)+(0cm,-6cm)$);

	\draw[line width=2pt,purple,latex-latex,,dash pattern=on 10pt off 3pt]($(current page.south west)+(0cm,10.7cm)$)--($(current page.south east)+(0cm,10.7cm)$);

	\draw[line width=2pt,purple,latex-latex,,dash pattern=on 10pt off 3pt]($(current page.south west)+(0cm,9.7cm)$)--($(current page.south east)+(0cm,9.7cm)$);

	\draw[line width=2pt,purple,latex-latex,,dash pattern=on 10pt off 3pt]($(current page.south west)+(0cm,0.7cm)$)--($(current page.south east)+(0cm,0.7cm)$);
\end{tikzpicture}

\begin{tikzpicture}[remember picture,overlay]
	\draw[opacity=0]($(current page.north)+(6.795cm,0cm)$)--($(current page.south)+(6.795cm,0cm)$);
	\draw[opacity=0]($(current page.north west)+(0.7cm,0cm)$)--($(current page.south west)+(0.7cm,0cm)$);
	\draw[opacity=0]($(current page.north west)+(0cm,-1.2cm)$)--($(current page.north east)+(0cm,-1.2cm)$);
	\draw[opacity=0]($(current page.north west)+(0cm,-5.5cm)$)--($(current page.north east)+(0cm,-5.5cm)$);
	\draw[opacity=0]($(current page.north west)+(0cm,-6cm)$)--($(current page.north east)+(0cm,-6cm)$);
	\draw[opacity=0]($(current page.south west)+(0cm,10.7cm)$)--($(current page.south east)+(0cm,10.7cm)$);
	\draw[opacity=0]($(current page.south west)+(0cm,9.7cm)$)--($(current page.south east)+(0cm,9.7cm)$);
	\draw[opacity=0]($(current page.south west)+(0cm,0.7cm)$)--($(current page.south east)+(0cm,0.7cm)$);
\end{tikzpicture}


\tikzexternalenable % Restaura la el precompilado de figuras.
\restoregeometry % Restaura los margenes del documento.
