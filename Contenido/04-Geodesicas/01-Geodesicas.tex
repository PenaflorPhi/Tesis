\section{Geodésicas}\label{Sección: Geodésicas}
En este punto tenemos las herramientas para poder hablar de propiedades geométricas en las variedades, en particular discutiremos a las curvas geodésicas. Si bien las geodésicas pueden ser definidas sobre variedades suaves en general, ya que dependen únicamente de la conexión afín con la cual dotemos a la variedad, nosotros consideraremos variedades Riemannianas, y, por lo tanto, consideraremos su conexión canónica, la conexión de Levi-Civita.

Entender lo que son y como calcular las curvas geodésicas es de gran utilidad en muchos problemas tanto de las matemáticas como la física y la ingeniería, ya que, su característica más importante es que, al menos de manera local, estas curvas minimizan distancias; en este sentido es que las geodésicas generalizan el concepto de línea recta a las variedades suaves.

Es evidente que, si queremos ser capaces de hablar una curva que minimice distancias, primero debemos definir como es que mediremos distancias. Para hacer esto aprovecharemos el hecho de que la métrica define un producto interno en el espacio tangente a cada punto de una variedad, esto nos permite dotar a cada espacio tangente con una norma.

\begin{definition}[Longitud de un vector tangente]
Sean $(M,g)$ una variedad Riemanniana y $p \in M$ un punto arbitrario. Sea $v \in T_{p}M$, definiremos la \textit{longitud de v} como:
\[
	\|v\| = \sqrt{g_{p}(v,v)}
\]
\end{definition}
Inmediatamente podemos notar que la longitud de los vectores tangentes está bien definida dado que la métrica es definida positiva; además, cuando consideramos la métrica euclidiana, está definición usual de longitud de un vector. De modo similar, tomando inspiración de como medimos la longitud de una curva en $\mathbb{R}^{n}$ podemos definir la longitud de una curva suave en una variedad Riemanniana.
\begin{definition}[Longitud de una curva]
	Sea $\gamma: I \to M$ una curva suave en una variedad Riemanniana. Definimos la \textit{longitud de $\gamma$} en un subintervalo $[a,b] \subset I$ como:
	\[
		\ell(\gamma) = \int_{a}^{b} \|\gamma'(t)\| dt
	\]
\end{definition}

\begin{definition}[Reparametrización de una curva]
	Sea $\gamma: I \to M$ una curva suave, diremos que la curva $\overline{\gamma}$ es una \textit{reparametrización de $\gamma$} si es de la forma $\overline{\gamma} = \gamma \circ \phi : J \to M$, donde $J \subset \mathbb{R}$ es un intervalo y $\phi: J \to I$ es un difeomorfismo.
\end{definition}

\begin{definition}[Curva regular y curva admisible]
	Sea $M$ una variedad suave y $\gamma: I \to M$ una curva suave, diremos que:
	\begin{itemize}
		\item $\gamma$ es una \textit{curva regular} si $\gamma'(t) \neq 0$ para cada $t \in I$.
		\item $\gamma$ es una curva regular por partes si existe una partición $(a_0, \ldots, a_k)$ de $I$ tal que $\gamma|_{[a_{i-1},a_i]}$ es una curva regular, para $i=1, \ldots, k$.
		\item $\gamma$ es una \textit{curva admisible} si para cualquier partición de $I$ y cada subintervalo de la partición, la restricción de $\gamma$ a dicho subintervalo es una curva regular
	\end{itemize}
\end{definition}

\begin{lemma}[Propiedades de la longitud de una curva]
	Supongamos que $(M,g)$ es una variedad Riemanniana y $\gamma: [a,b] \to M$ una curva admisible en $M$. La longitud de $\gamma$ cumple las siguientes propiedades:
	\begin{enumerate}
		\item Si $a < c < b$, entonces:
		      \[
			      \ell(\gamma) = \ell(\gamma|_{[a,c]}) + \ell(\gamma|_{[c,b]})
		      \]
		\item Si $\overline{\gamma}$ es una reparametrización de $\gamma$, entonces:
		      \[
			      \ell(\gamma) = \ell(\overline{\gamma})
		      \]
		\item Si $(M,g)$ y $(\overline{M}, \overline{g})$ son variedades Riemannianas y $\phi: M \to \overline{M}$ es una isometría local, entonces:
		      \[
			      \ell_{g}(\gamma) = \ell_{\overline{g}}(\phi \circ \gamma)
		      \]
	\end{enumerate}
\end{lemma}

\begin{proof}\phantom{ }
	\begin{enumerate}
		\item Esta primera propiedad es una consecuencia de las propiedades elementales de la integral.
		      \begin{align*}
			      \ell(\gamma)
			       & = \int_{a}^{b} \|\gamma'(t)\|dt
			      = \int_{a}^{c} \|\gamma'(t)\| dt \int_{c}^{b}\|\gamma'(t)\|dt \\
			       & = \ell(\gamma|_{[a,c]}) + \ell(\gamma|_{[c,b]})
		      \end{align*}
		\item Esta segunda propiedad es una consecuencia de la regla de la cadena y del teorema fundamental del cálculo, tendremos:
		      \begin{align*}
			      \ell(\overline{\gamma})
			       & = \int_{a}^{b} \|\overline{\gamma}'(t)\| dt
			      = \int_{a}^{b} \|(\gamma \circ \phi)'(t) \| dt                \\
			       & = \int_{a}^{b} \|\gamma'(\phi(t))\| \phi'(t) dt            \\
			       & = \int_{\phi(a)}^{\phi(b)} \|\gamma(t)\| dt = \ell(\gamma)
		      \end{align*}
		\item Aplicando directamente la definición de isometría tendremos que, si $(M,g)$ y $(\overline{M}, \overline{g})$ son  variedades localmente isométricas, entonces, para cada $p \in M$ y cada $v \in T_{p}M$
		      \[
			      \|v\|_{g} = \sqrt{g(v,v)}
			      = \sqrt{\overline{g}(F(v),F(v))} = \|F(v)\|_{\overline{g}},
		      \]
		      se sigue que:
		      \begin{align*}
			      \ell(\gamma) = \int_{a}^{b} \|\gamma'(t)\|_g dt
			      = \int_{a}^{b} \|(F \circ \gamma)'(t)\| dt = \ell(F \circ \gamma)
		      \end{align*}
	\end{enumerate}
\end{proof}

\begin{definition}[Curvas por longitud de arco]
	Sea $\gamma: [a,b] \to M$ una curva admisible. Definiremos la \textit{función de longitud de arco} de la curva $\gamma$ como la función $s: [a,b] \to \mathbb{R}^{+}$ dada por:
	\[
		s(t) = \int_{a}^{t} \|\gamma'(u) \| du.
	\]
	Del teorema fundamental del cálculo se puede deducir que $s'(t) = \| \gamma'(t) \|$. Como se vio en la subsección \ref{Subsección: Curvas En Variedades}, podemos interpretar a $\gamma'(t)$ como la velocidad de la curva en cada punto, siguiendo con esta idea podemos llamarle al escalar dado por $\|\gamma'(t)\|$ la \textit{rapidez} de $\gamma$ en $t$.

	Diremos que la curva $\gamma$ tiene \textit{rapidez unitaria} si $\|\gamma'(t)\| = 1$ para cada $t \in I$. Es evidente que si la curva tiene rapidez unitaria la función de longitud de arco quedaría como:
	\[
		s(t) = \int_{a}^{t} \|\gamma'(u)\|du = t - a.
	\]
	Si $\gamma$ es una curva con rapidez unitaria en un intervalo de la forma $[0,b]$, entonces, la función de longitud de arco tendrá una forma aún más simple,
	\[
		s(t) = \int_{0}^{t} \|\gamma'(u)\| du = t,
	\]
	decimos que una curva con rapidez unitaria en un intervalo de la forma $[0,b]$ es una \textit{curva parametrizada por longitud de arco}.
\end{definition}

\begin{lemma}
	Sea $(M,g)$ una variedad Riemanniana. Cada cura regular en $M$ tiene una reparametrización con rapidez unitaria.
\end{lemma}

\begin{proof}
	Sea $\gamma: I \to M$ una curva regular. Eligiendo algún punto arbitrario $t_0 \in I$ definimos la función $s: I \to \mathbb{R}$ como:
	\[
		s(t) = \int_{t_0}^{t} \|\gamma'(u)\|du
	\]
	tendremos que $s$ es una función estrictamente creciente, por el teorema de la función inversa podemos garantizar que $s$ es un difeomorfismo local de $I$ a un intervalo $J \subset \mathbb{R}$. Definiremos la curva $\overline{\gamma} = \gamma \circ s^{-1}: J \to \mathbb{R}$, esta será una reparametrización estrictamente creciente de $\gamma$, además, tiene rapidez unitaria, esto se sigue, nuevamente, del teorema de la función inversa, y del hecho que $s'(t) = \|\gamma'(t)\|$
	\[
		\|\gamma'(t)\| = \|(\gamma \circ s^{-1})'(t)\| = \frac{ \|\gamma'(s^{-1}(t))\| }{\| s'(s^{-1}(t))\| } = 1
	\]
\end{proof}

\begin{lemma}
	Sea $(M,g)$ una variedad Riemanniana. Cada curva admisible en $M$ tiene una única reparametrización por longitud de arco con rapidez unitaria.
\end{lemma}

\begin{proof}
	Sea $\gamma: [a,b] \to M$ una curva admisible. Procederemos por inducción sobre el número de segmentos de las particiones admisibles. Si $\gamma$ tiene un único segmente, entonces por definición será una curva regular; por el lema anterior, existe una reparametrización $\overline{\gamma}: I \to M$ tal que $\overline{\gamma}$ tiene rapidez unitaria y, tomando $I$ como el intervalo $[t_0, b]$ la curva estará parametrizada por longitud de arco.

	Supongamos ahora que si el intervalo $[a,b]$ tiene una partición con $k$ segmentos admisibles, también existe una reparametrización por longitud de arco en con velocidad unitaria.

	Supongamos que $[a,b]$ tiene una partición admisible $(a_0, \ldots, a_{k+1})$ con $k+1$ segmentos. Por la hipótesis de inducción existirán homeomorfismos $\phi : [0,c] \to [a, a_{k}]$ y $\psi: [0,d] \to [a_k, b]$ tales que $\gamma \circ \phi$ y $\gamma \circ \psi$ son reparametrización por longitud de arco con velocidad unitaria para $\gamma|_{[a_0,a_k]}$ y $\gamma|_{[a_k,b]}$ respectivamente, sabiendo esto podemos definir la función $\overline{\gamma}: [0,d] \to [a,d]$ como:
	\[
		\overline{\gamma} = \begin{cases}
			\gamma \circ \phi (s),   & s \in [0,c]      \\
			\gamma \circ \psi (s-c), & s \in [c, c + d] \\
		\end{cases}
	\]
	es evidente, por cómo se ha definido la función $\overline{\gamma}$ que esta es una reparametrización por longitud de arco de $\gamma$ la cual tiene velocidad unitaria.

	Para probar la unicidad supongamos que $\overline{\gamma} = \gamma \circ \phi$ y $\hat{\gamma} = \gamma \circ \psi$ son reparametrizaciones por longitud de arco para la curva $\gamma$ y que ambas tienen velocidad unitaria. Dado que tanto $\overline{\gamma}$ como $\hat{\gamma}$ tiene la misma longitud, $\phi$ y $\psi$ deberán estar definidas en el mismo dominio $[0,c]$, por lo cual ambas funciones serán homeomorfismos por partes de $[0,c]$ a $[a,b]$. Definamos la función $\lambda = \phi^{-1} \circ \psi : [0,c] \to [0,c]$, $\lambda$ es un homeomorfismo que satisface $\hat{\gamma} = \gamma \circ \phi \circ \lambda = \overline{\gamma} \circ \lambda$. Por hipótesis $\overline{\gamma}$ y $\hat{\gamma}$ son curvas admisibles, por lo cual serán regulares excepto quizá en un conjunto de medida cero, por lo cual, utilizando la regla de la cadena se sigue que para cada $s \in [0,c]$:
	\[
		1 = \| \hat{\gamma}(s)\| = \|\overline{\gamma}(\lambda(s)) \lambda'(s)\| = \|\gamma'(\lambda(s)) \| \| \lambda'(s) \| = \lambda'(s),
	\]
	Al ser $\lambda$ un homeomorfismo será continuo y $\lambda(0) = 0$, por lo cual $\lambda(s) = s$, así, podemos concluir que $\overline{\gamma} = \hat{\gamma}$.
\end{proof}

\begin{lemma}
	Cualesquiera dos puntos en una variedad suave pueden ser conectados por una curva admisible.
\end{lemma}

\begin{proof}
	Es evidente que las variedades (topológicas) son tanto conexas como conexas por caminos, esto dado que ambas propiedades se preservan bajo homeomorfismos y, como se muestra en el anexo \ref{Anexo: Topologia De Variedades} las bolas coordenadas forman una base para la topología de la variedad.

	Sean $p, q \in M$ y sea $\gamma: [a,b] \to M$ una curva que conecte a $p$ y $q$. Por la compacidad de $M$ se tendrá que existe una partición de $[a,b]$ para la cual la imagen de $\gamma([a_{i-1}, a_i])$ esté contenida en una única bola coordenada para cada $i$.

	Podemos remplazar la curva $\gamma$ por la preimagen de una recta en $\mathbb{R}^{n}$ que coincida con la curva en la frontera de la bola coordenada, considerado la unión de todas estas rectas obtendremos una curva admisible que conectará a los puntos $p$ y $q$.
\end{proof}

\begin{definition}[Distancia Riemanniana]
	Sea $(M,g)$ una variedad Riemanniana. Dados dos puntos $p,q \in M$ la \textit{distancia Riemanniana} se define como el escalar:
	\[
		d(p,q) = \inf_{\gamma} \{\ell(\gamma): \gamma: [a,b] \to M, \gamma(a) = p, \gamma(b) = q\},
	\]
	donde $\gamma$ es una curva admisible en $M$. A esta curva se le conoce como una \textit{curva minimizante}.
\end{definition}

\begin{definition}[Geodésica]
	Sea $M$ una variedad Riemanniana. Decimos que una curva $\gamma: I \to M$ es una \textit{geodésica} si, para cada $t \in I$ se tiene que:
	\[
		\frac{D\gamma'(t)}{dt} = 0
	\]
\end{definition}
Las geodésicas como mencionábamos son curvas que, al menos de manera local, minimizan distancias. Enunciaremos dos teoremas que caracterizan esta propiedad, sin embargo, no las demostraremos. (El margen de esta tesis es demasiado pequeño para contener la demostración).

\begin{theorem}
	Sea $(M,g)$ una variedad Riemanniana. Toda curva minimizante en $M$ es una geodésica cuando se le da una parametrización con rapidez unitaria.
\end{theorem}

\begin{theorem}
	Sea $M$ una variedad Riemannian, $p \in M$ algún punto arbitrario y $B_{p}$ una bola centrada en $p$. Sea $\gamma: [a,b] \to B_{p}$ una geodésica (o la restricción de una geodésica), con $\gamma(a) = p$, además denotaremos a $\gamma(b)$ por $q$.

	Si $\alpha: [a,b] \to M$ es cualquier curva admisible que una a los puntos $p$ y $q$, entonces:
	\[
		\ell(\gamma) \leq \ell(\alpha),
	\]
	y, si se da el caso que $\ell(\gamma) = \ell(\alpha)$, entonces se tendrá que $\gamma([a,b]) = \alpha([a,b])$.
\end{theorem}

Estos dos resultados nos están diciendo que la distancia más corta entre dos puntos es una geodésica, sin embargo, el segundo teorema también nos está diciendo que esta es una propiedad local. Do \textcite{do1992riemannian} da un ejemplo de que esta propiedad es local en la esfera, dado un punto $p$ en la esfera, las geodésicas dejan de ser minimizantes después de la antípoda de $p$.

Un resultado si mostraremos sobre las geodésicas es su existencia y unicidad, ya que, este resultado nos dará una manera explícita para calcular a las geodésicas.

\begin{theorem}[Existencia y unicidad de geodésicas]
	Sea $M$ una variedad suave equipada con una conexión afín $\nabla$. Para cada punto $p \in M$, cada vector tangente $v \in T_{p}M$ y cada real $t_0 \in \mathbb{R}$ existe un intervalo abierto $I \subset \mathbb{R}$ que contiene a $t_0$ y una geodésica $\gamma: I \to M$ que satisface $\gamma(t_0) = p$ y $\gamma'(t_0) = v$. Cualesquiera geodésicas que satisfagan las misma condiciones coincidirán en la intersección de sus dominios.
\end{theorem}

\begin{proof}
	Sea $(U,\phi)$ una carta suave en $M$ que contenga a $p$. Expresaremos a la geodésica $\gamma$ en las coordenadas locales de la carta como $\gamma(t) = (\gamma_{1}(t), \ldots, \gamma_{n}(t))$. Por definición de geodésica sabemos que para cada $t \in I$ se tiene:
	\[
		\frac{D \gamma'(t)}{dt} = 0,
	\]
	en los términos estudiados en la sección anterior, esto quiere decir que una geodésica es una curva cuyo campo de velocidades es siempre paralelo. Utilizando la expresión encontrada en el teorema \ref{Teorema: Existencia y Unicidad de Campos Paralelos} podemos representar, en términos de las coordenadas locales de $U$, al campo de velocidades, obteniendo la expresión:
	\begin{align*}
		\frac{D\gamma(t)}{dt} & = \sum_{k=1}^{n} \left(
		\frac{d\gamma'_k(t)}{dt} + \sum_{i=1}^{n}\sum_{j=1}^{n} \frac{d\gamma_{i}(t)}{dt} \Gamma_{ij}^{k}\gamma_{j}'(t)
		\right) \partial_{k}                                                                                                                    \\
		                      & = \sum_{k=1}^{n} \left( \gamma''(t) + \sum_{i=1}^{n}\sum_{j=1}^{n} \Gamma_{ij}^{k} \gamma_{i}'(t)\gamma_{j}'(t)
		\right)\partial_{k}
	\end{align*}
	Luego, este campo será nulo para cada $t \in I$ si y solo si:
	\[
		\gamma''(t) + \sum_{i=1}^{n}\sum_{j=1}^{n} \Gamma_{ij}^{k}\gamma_{i}'(t) \gamma_{j}'(t) = 0
	\]
	para cada $1 \leq k \leq n$. De este modo tendremos un sistema de ecuaciones diferenciales cuasilineales de segundo orden. Podemos reducir este sistema realizando la sustitución por la variable auxiliar $\gamma_{i}'(t) = x_{i}'(t)$, obteniendo el sistema de ecuaciones diferenciales ordinarias de primer orden:
	\[
		x_{k}'(t) + \sum_{i=1}^{n} \sum_{j=1}^{n} \Gamma_{ij}^{k} x_{i}(t) x_{j}(t) = 0.
	\]
	Este sistema de ecuaciones tiene solución y dicha solución es única, esto está garantizado por el Teorema de Picard-Lindelöf.
\end{proof}

A la ecuación sin reducir usualmente se le conoce como la \textit{ecuación geodésica}. A continuación, resolveremos la ecuación para ver cómo podemos encontrar las geodésicas de diferentes variedades.
