\chapter{Conclusiones}
Nuestro principal objetivo con este trabajo fue describir las curvas geodésicas en el semiplano superior de Poincaré. Para lograr este objetivo desarrollamos algunos conceptos básicos de la teoría de variedades, como lo son las variedades topológicas, las variedades suaves y las variedades Riemannianas. 

Para poder desarrollar estos conceptos y llegar a nuestro objetivo fue necesario hacer uso de herramientas de la geometría diferencial, trasladando conceptos del cálculo en espacios euclidianos a las variedades suaves. Estas herramientas, desarrolladas a lo largo del capítulo \ref{Capítulo: Cálculo en Variedades}, son las que nos permitieron definir los conceptos de métrica Riemanniana, conexiones y finalmente las curvas geodésicas. 

Las ideas desarrolladas en esta tesis y los resultados obtenidos son de gran utilidad en una multitud de áreas de las matemáticas, quizá los resultados más inmediatos que se podrían obtener a partir de lo estudiado en este trabajo son aquellos relacionados con la curvatura de las variedades, como el \textit{Teorema de Gauss$-$Bonnet}, el cual nos da una relación entre la curvatura de una variedad y su topología. También es posible hablar de \textit{Grupos de Lie}, estos son grupos los cuales también son variedades suaves; los grupos de Lie han demostrado ser una herramienta indispensable para el estudio de las ecuaciones diferenciales. 

Además de esto, las ideas presentadas en este trabajo también pueden ayudarnos a comenzar a entender problemas importantes, como la \textit{Conjetura de Poincaré}, resuelta por Grigori Perelman, la conjetura es planteada en términos de variedades suaves y para su solución fue necesario hacer uso de una herramienta de la geometría diferencial conocida como \textit{flujo de Ricci}. Otro problema importante es el \textit{Último Teorema de Fermat}, demostrado por Andrew Wiles, para realizar la demostración de dicho teorema Wiles hizo uso de formas modulares, las cuales son funciones con propiedades muy particulares definidas sobre el semiplano superior. 
