\begin{example}[Geodésicas en $\mathbb{S}^{2}$]
  Si consideramos a la $2-$esfera como una superficie de revolución y la dotamos con la métrica inducida por $\mathbb{R}^{3}$ como se hizo en el ejemplo \ref{Ejemplo: Métrica en la Esfera} es posible demostrar que las geodésicas son segmentos de los círculos máximos, los cuales dividen a la esfera en dos hemisferios.

  Los cálculos para encontrar las geodésicas pueden ser consultados en \textcite{delia2011geodesia}.
\end{example}

