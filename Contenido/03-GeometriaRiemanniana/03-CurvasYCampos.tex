\subsection{Conexiones, Campos y Curvas}
\label{Subsección: Conexiones Campos y Curvas}

Ahora que tenemos esta nueva derivada nos gustaría poder utilizarla para estudiar el comportamiento de curvas suaves en variedades.

\begin{definition}
	Sean $M$ una variedad suave, $I \subset \mathbb{R}$ un intervalo y $\gamma: I \to M$ una curva suave en $M$. Diremos que $V: I \to TM$ es un \textit{campo vectorial a lo largo de $\gamma$} si $V(t) \in T_{\gamma(t)}M$ para cada $t \in I$.
\end{definition}

Notemos que para cualquier curva suave $\gamma: I \to M$ siempre es posible encontrar un campo vectorial a lo largo de ella, ya que cualquier campo suave $X \in \mathfrak{X}(M)$ nos induce un campo $V$ a lo largo de $\gamma$, esto simplemente definiendo:
\[
	V = X \circ \gamma
\]
Si $V$ es un campo vectorial a lo largo de una curva $\gamma: I \to M$, entonces, para cada $t_{0} \in I$ es posible expresar a $V(t)$ en una vecindad de $\gamma(t)$ como:
\[
	V(t_{0}) = \sum_{i=1}^{n} V_{i}(t_{0}) \partial_{i} |_{t_{0}}
\]

\begin{lemma}
	Sea $M$ una variedad suave equipada con una conexión $\nabla$. Existe una única correspondencia que asocia al campo vectorial $V$ a lo largo de una curva $\gamma: I \to M$, otro campo vectorial suave $\frac{DV}{dt}$ a lo largo de $\gamma$, tal que:
	\begin{enumerate}
		\item $\frac{D}{dt}(V+W) = \frac{DV}{dt}  + \frac{DW}{dt}$
		\item $\frac{D}{dt}(fV)  = \frac{df}{dt}V + f\frac{DV}{dt}$,
	\end{enumerate}
	donde $V$ y $W$ son campos vectoriales suaves a lo largo de $\gamma$ y $f$ es una función suave en $I$.
	\begin{enumerate}
		\item[3.] Si $V$ es inducido por un campo vectorial $X \in \mathfrak{X}(M)$, entonces:
			\[
				\frac{DV}{dt} = \nabla_{\gamma'(t)}X.
			\]
	\end{enumerate}
	A este campo vectorial $\frac{DV}{dt}$ lo llamamos la \textit{derivada covariante de $V$ a lo largo de $\gamma$}.
\end{lemma}

\begin{proof}
	Supongamos que existe una asignación que cumple todas las propiedades del lema. Sea $V$ un campo vectorial suave a lo largo de una curva $\gamma: I \to M$, como hemos mencionado anteriormente, dado un punto $t_{0} \in I$ podemos expresar a $V$ en una vecindad de $\gamma(t_{0})$ como:
	\[
		V = \sum_{j=1}^{n} V_{j}(t_{0}) \partial_{j}|_{t_{0}},
	\]
	luego, por la primera y segunda propiedad de la asignación se tendrá la siguiente cadena de igualdades:
	\begin{align*}
		\left. \frac{DV}{dt} \right|_{t_{0}}
		 & = \frac{D}{dt} \left( \sum_{j=1}^{n} V_{j}(t_{0}) \partial_{j}|_{t_{0}} \right) \\
		 & = \sum_{j=1}^{n} \frac{D}{dt} ( V_{j}(t_{0}) \partial_{j}|_{t_{0}})             \\
		 & = \sum_{j=1}^{n} \frac{dV_{j}(t_{0})}{dt} \partial_{j}(t_{0})
		+ \sum_{j=1}^{n} V_{i}(t_{0}) \left. \frac{D\partial_{j}}{dt} \right|_{t_{0}}
	\end{align*}
	Gracias al lema \ref{Lema - Conexión Afín en Coordenadas} sabemos que el valor de la conexión afín $(\nabla_{X}Y)(p)$ depende únicamente de los valores del vector tangente $ X(p) $ y el campo $ Y $ a lo largo de una curva $\alpha: I \to M$ que satisfaga que $\alpha(0) = t$ y $\alpha'(0) = X(p)$, por lo cual podemos interpretar la tercera propiedad del lema como:
	\[
		\nabla_{\gamma'(t_{0})}\partial_{j} = \nabla_{X}Y (\gamma(t_{0})),
	\]
	donde $X$ es cualquier campo suave en $M$ que satisface $X(\gamma(t_{0})) = \gamma'(t_{0})$. Por lo cual, utilizando la tercera propiedad del lema y la primera propiedad de la definición de conexión afín se obtienen las igualdades:
	\begin{align*}
		\left. \frac{D\partial_{j}}{dt} \right|_{t_{0}}
		 & = (\nabla_{\gamma'(t)}\partial_{j})(t_{0})                                                                 \\
		 & = \left(
		\nabla_{\sum_{i=1}^{n} \left. \frac{d\gamma_{i}}{dt} \right|_{t_{0}}  \partial_{i}} \partial_{j}
		\right) (t_{0})                                                                                               \\
		 & = \sum_{i=1}^{n} \left. \frac{d\gamma_{i}}{dt} \right|_{t_{0}} (\nabla_{\partial_{i}} \partial_{j})(t_{0})
	\end{align*}

	Por lo tanto, podemos expresar al campo $\frac{DV}{dt}$ de manera local como:
	\[
		\left. \frac{DV}{dt} \right|_{t_{0}} =
		\sum_{k=1}^{n} \left(
		\left. \frac{dV_{k}}{dt} \right|_{t_{0}} +
		\sum_{i=1}^{n}\sum_{j=1}^{n} \left. \frac{d\gamma_{i}}{dt} \right|_{t_{0}}
		\Gamma_{ij}^{k}(t_{0}) V_{j}(t_{0})
		\right) \partial_{k}|_{t_0}
	\]
	Esto basta para probar la unicidad, dado que si existiera otra asignación que cumpliera las tres propiedades del lema esta debería tener exactamente la misma forma. Más aún, por cómo se ha construido es fácil verificar que $\frac{DV}{dt}$ es un campo vectorial que cumple las propiedades del lema, con:
	\[
		\frac{DV}{dt} = \sum_{k=1}^{n} \left(\frac{dV_{k}}{dt}
		+ \sum_{i=1}^{n}\sum_{j=1}^{n} \frac{d\gamma_{i}}{dt} \Gamma_{ij}^{k} V_{j}\right) \partial_{k}
	\]
\end{proof}

\begin{definition}[Campo Vectorial Paralelo]
	Sea $M$ una variedad suave equipada con una conexión afín $\nabla$. Diremos que un campo vectorial a lo largo de una curva $\gamma: I \to M$ es un \textit{campo paralelo} si:
	\[
		\frac{DV}{dt} \equiv 0
	\]
\end{definition}

\begin{theorem}\label{Teorema: Existencia y Unicidad de Campos Paralelos}
	Sea $M$ una variedad suave equipada con una conexión afín $\nabla$. Sea $\gamma: I \to M$ una curva suave y sea $V_{t_{0}}$ un vector tangente a $M$ en $\gamma(t_{0})$, con $t_{0} \in I$. Entonces, existe un único campo vectorial paralelo $V$ a lo largo de $\gamma$, para el cual $V_{t_{0}} = V(t_{0})$.
\end{theorem}

\begin{proof}
	Comencemos suponiendo que la imagen de $\gamma$, $\gamma(I)$, está contenido en alguna carta coordenada. Si este fuera el caso, como se mostró en el lema anterior, podemos representar al campo $\frac{DV}{dt}$ en las coordenadas locales de la carta como:
	\[
		\frac{DV}{dt} = \sum_{k=1}^{n} \left(\frac{dV_{k}}{dt}
		+ \sum_{i=1}^{n}\sum_{j=1}^{n} \frac{d\gamma_{i}}{dt} \Gamma_{ij}^{k} V_{j}\right) \partial_{k},
	\]
	por lo cual, la igualdad $\frac{dV}{dt} = 0$ se obtendrá si y solo si:
	\[
		\frac{d V_{k}}{dt} + \sum_{i=1}^{n}\sum_{j=1}^{n} \frac{d\gamma_{i}}{dt} \Gamma_{ij}^{k} V_{j} = 0
	\]
	para cada $1 \leq k \leq n$, con condiciones iniciales $V_{1}(t_{0}) = V^{1}_{t_{0}}, \ldots, V_{n}(t_{0}) = V^{n}_{t_0}$.

	El teorema de Picard-Lindelöf nos permite garantizar que este sistema de ecuaciones diferenciales tiene solución y que dicha solución es única. Si, ocurre que $\gamma(I)$ no está contenido en una carta coordenada entonces debe ocurrir que $\gamma(I)$ sea un conjunto compacto, por lo cual podrá ser cubierto por una cantidad de cartas coordenadas, en cada una de las cuales podemos aplicar el argumento anterior para garantizar la existencia y unicidad de las soluciones para los sistemas de ecuaciones.
\end{proof}
