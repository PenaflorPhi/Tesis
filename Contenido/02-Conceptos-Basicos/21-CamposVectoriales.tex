\section{Campos Vectoriales}\label{Sección: Campos Vectoriales}
Siguiendo con la extensión y generalización que hemos estado realizando de conceptos conocidos de cálculo en espacios euclidianos, ahora extenderemos la idea de los campos vectoriales, estos objetos nos darán una forma de asignar a cada punto de una variedad un vector del espacio tangente.

Para poder definir lo que es un campo vectorial primero hablaremos de lo que son los fibrados vectoriales, lo cual responderá la pregunta porque hemos decido llamar fibrados a la colección de espacios tangentes a una variedad, y qué es una fibra.

\begin{definition}[Fibrado Vectorial]
 Sean $E$ y $M$ variedades suaves, $\pi: E \to M$ un mapa suave y sobreyectivo, al cual llamaremos la \it{proyección fibrada}, diremos que $\pi$ es \it{localmente trivial de rango $k$} si:
 
\begin{itemize}
\item Para cada $p \in M$, la preimagen $\pi^{-1}(\{p\})$ tiene estructura de  espacio vectorial $k-$dimensional, llamaremos a esta preimagen la \it{fibra en $p$} y usualmente se denotará como $E_p$.

\item Para cada $p \in M$ existe una vecindad $U$ que lo contiene y $\phi: U \times \R^{k} \to \pi^{-1}(U)$ un difeomorfismo tal que para cada $q \in U$:
\begin{itemize}
    \item $(\pi \circ \phi)(q,v) = q$, para cada $v \in \R^{k}$.
    \item El mapeo $v \mapsto \phi(q,v)$ es un isomorfismo lineal entre $\R^{k}$ y $\pi^{-1}(q)$.
\end{itemize}
\end{itemize}

  Diremos que $U$ es un \it{conjunto abierto trivializante} en $E$. La colección $\{(U_\alpha, \phi_\alpha)\}_{\alpha}$, donde $\{U_{\alpha}\}_{\alpha}$ es una cubierta abierta de $M$ es llamada la trivialización local de $E$, y la cubierta abierta $\{U_\alpha\}_{\alpha}$ es llamada la cubierta abierta trivializante de $M$ para $E$.

  Un \it{fibrado vectorial suave de rango $k$} es una terna $(E,M,\pi)$ donde $E$ y $M$ son variedades suaves y $\pi: E \to M$ es un mapa suave y sobreyectivo que es localmente trivial de rango $k$. A la variedad $E$ la llamamos el \it{espacio total} del fibrado vectorial y a la variedad $M$ el \it{espacio base}. Por simplicidad se dice que $E$ es un fibrado vectorial sobre $M$.
\end{definition}

Por los resultados mostrados en la sección anterior es evidente que si $M$ es una variedad suave, la terna $(TM, M, \pi)$, donde $TM$ es el fibrado tangente y $\pi$ es la proyección natural de $TM$ sobre $M$ será un fibrado vectorial suave.

\begin{definition}[Sección de un Fibrado Vectorial]
  Una \it{sección} de un fibrado vectorial $\pi: E \to M$ es un mapa suave $\sigma: M \to E$ tal que $\pi \circ \sigma = \id_{M}$, esto quiere decir que para cada $p \in M$, $\sigma(p) \in E_p$. Diremos que es una \it{sección suave} si el mapa $\sigma$ es suave de $M$ a $E$. Denotaremos al conjunto de todas las secciones suaves de $E$ como $\Gamma(E)$.
\end{definition}

\begin{definition}[Campo Vectorial]
  Un \it{campo vectorial} $X$ en una variedad $M$ es una sección del fibrado tangente $\pi: TM \to M$, esto es, $X: M \to TM$ es un mapa tal que $X(p) \in T_{p}(M)$ para cada $p \in M$. Además diremos que que es un \it{campo vectorial suave} si $X$ es un mapa suave. Denotaremos al conjunto formado por todas los campos vectoriales suaves en $M$ como $\mathfrak{X}(M)$
\end{definition}

\begin{example}
  Sea $M$ una variedad suave y $(U,\phi)=(U,\phi_1, \dots,\phi_n)$ una carta suave sobre $M$. La asignación:
  \[
    \frac{\partial}{\partial \phi_i}: p \mapsto \left. \frac{\partial}{\partial \phi_i}\right|_{p}
  \]
  nos da un campo vectorial en $U$. A la $n$-tupla ordenada $(\frac{\partial}{\partial \phi_1}, \dots, \frac{\partial}{\partial \phi_n})$ le llamamos un \it{marco local}, como hemos visto el marco local forma una base de $T_p(M)$ para $p \in U$, además, si $X$ es un campo vectorial suave definido en un conjunto que incluya a $U$, entonces existirán funciones suaves $X_i$ definidas en $U$ tales que:
\[
  X(p) = \sum_{i=1}^{n}X_i(p)\left.\frac{\partial}{\partial\phi_i} \right|_{p}
\]
Este ejemplo es muy importante, ya que nos está diciendo que los campos vectoriales forman una base en cada punto de una variedad para el espacio tangente 
\end{example}

Por el lema \ref{Lemma: Espacio Tangente a Subvariedad}, si $M$
 es una variedad suave $U \subset M$ es una subvariedad abierta, entonces para cada $p \in U$ podemos identificar al espacio tangente $T_p(U)$ con el el espacio tangente $T_p(M)$, por lo que si $X$ es un campo suave en $M$ y $U \subset M$ es abierto, la restricción $X|_{U}$ será un campo suave.

\begin{definition}[Campo Vectorial a lo Largo de un Conjunto]
  Si $M$ es una variedad suave y $A \subseteq M$ es un subconjunto de $M$, no necesariamente abierto. Diremos que $X: A \to TM$ es un \it{campo vectorial a lo largo de $A$} si $X$ es continuo y satisface $\pi \circ X = \id_{A}$. Diremos que $X$ es un \it{campo vectorial suave a lo largo de $A$} si para cada $p \in A$ existe una vecindad $V_p \subseteq M$ y un campo vectorial $\hat{X}$ en $V_p$ que coincide con $X$ en $V \cap A$.
\end{definition}

\begin{lemma}
  Sea $M$ una variedad suave y sea $A \subset M$ un subconjunto cerrado. Supongamos que $X$ es un campo vectorial suave a lo largo de $A$. Dado un subconjunto $U$ abierto que contenga a $A$, existirá un campo vectorial global $\hat{X}$ en $M$ tal que $\hat{X}|_{A} = X$ y $\sup(\hat{X}) \subseteq U$.
\end{lemma}

\begin{proof}
  Sea $\{(V_\alpha,\psi_\alpha)\}$ un atlas suave en $M$ formado por bolas precompactas. $\mathcal{V}=\{V_\alpha\}$ es una cubierta abierta en $M$ por lo que cada $p \in A$ estará contenida en algún $V_\alpha$, y $\psi(p) = (x_1,\dots,x_n) \in \R^n$. Definiremos las funciones $X_\alpha: \R^n \to TM$ como:
  \[
    X_{\alpha}(\psi(p)) = \begin{cases}
      X(p), & p \in A\\
      0, & p \notin A
    \end{cases}
  \]

  Luego, por el teorema \ref{Teorema: Existencia de Particiones Suave de la Unidad} podemos garantizar que existirán particiones suaves de la unidad $\hat{f_\alpha}$ subordinada al atlas $\mathcal{V}$. Definimos el mapa $\hat{X}: M \to TM$ como:
  \[
    \hat{X} = \sum_{\alpha} X_{\alpha}f_\alpha
  \]

  Está suma convergerá dado que será diferente de cero solo en un número finito de puntos por ser la partición de la unidad localmente finita, además el lema \ref{Lemma: Lema de Extensión para Funciones Suaves} garantiza que para cada conjunto abierto $U$ que contengan a $A$ existirán funciones $\hat{f}$ tales que $\hat{X} = \sum_\alpha  X_\alpha f_\alpha$ coincide con $X$ en $A$ y $\sup(\hat{X}) \subseteq U$. 
\end{proof}

\begin{theorem}
  Sea $(E,M, \pi)$ un fibrado vectorial suave, $\sigma, \tau \in \Gamma(E)$ y sea $f \in C^{\infty}(M)$. Si definimos la suma suma de secciones suaves y el producto por una función como sigue:
\begin{align*}
  (\sigma + \tau)(p) &= \sigma(p) + \tau(p), \quad p \in M\\
  (f\sigma)(p) &= f(p)\sigma(p), \quad p \in M
\end{align*}
  Entonces $\Gamma(E)$ es un módulo sobre el anillo $C^{\infty}(M)$.
\end{theorem}

\begin{proof}
  Dado que tanto $\sigma$ como $\tau$ son secciones tendremos que elegido un punto $p \in M$, $\sigma(p)$ y $\tau(p)$ pertenecen a la fibra $E_p$, y como $E_p$ tiene estructura de espacio vectorial se tiene que $(\sigma + \tau)(p) \in E_p$, por lo tanto $\sigma + \tau$ es una sección del fibrado. Para mostrar la suavidad tomamos un punto $p \in M$ y un conjunto abierto trivializante en $E$ que contenga a $p$ con una trivialización suave 
  \[
    \phi: \pi^{-1}(U) \to U \times \R^k
  \]
Supongamos que para $q \in M$ se tiene:
\begin{align*}
  \phi \circ \sigma(q) &= (q, s_1(q), \dots, s_k(q))\\
  \phi \circ \tau(q) &= (q, t_1(q), \dots, t_k(q))
\end{align*}

  Como $\sigma$ y $\tau$ son mapas suaves, $\{s_i\}_{i=1}^k$ y  $\{t_i\}_{i=1}^k$ serán funciones suaves, y por definición $\phi$ es isomorfismo lineal, por lo que
\[
  \phi \circ (\sigma + \tau)(q) = (q, (s_1 + t_1)(q), \dots, (s_k + t_k)(q))
\]

  Por lo tanto $\sigma + \tau$ es un mapa suave en cada punto de $U$, en particular será suave en $p$. Por lo tanto $\sigma + \tau$ es una sección suave.

  De modo similar, $f\sigma$ será una sección del fibrado vectorial dado que al elegir un punto $p \in M$ se tiene que por definición del producto que $(f \circ \sigma)(p) = f(p)\sigma(p)$ y dado que $E_p$ tiene estructura de espacio vectorial y $f(p)$ es una simple constante $f(p)\sigma(p)$ pertenecerá a la fibra $E_p$. 

  Ahora tomemos un punto $p \in M$ y un conjunto $U$ trivializante en $E$ que contenga a $p$ con trivialización suave
  \[
    \phi: \pi^{-1}(U) \to U \times \R^{k}
  \]
  Supongamos que para $q \in M$ se tiene:
  \[
    \phi \circ \sigma(q) = (q, s_1(q), \dots, s_k(q))
  \]

  Como se ha mencionado anteriormente, al ser $\sigma(q)$ un mapa suave, cada $\{s_i\}_{i=1}^k$ será una función suave en $M$. Por la linealidad e $\phi$ se tendrá que 
  \[
    \phi \circ (f \sigma)(q) = (q, fs_1(q),\dots, fs_k(q))
  \]
  
  Por lo tanto cada una de las componentes es suave, así se garantiza que el producto definido de está manera es una sección suave. Aún más $\Gamma(E)$ será tanto un espacio vectorial como un módulo sobre el anillo $C^{\infty}(M)$.
\end{proof}

Este teorema también nos implica que el conjunto de campos vectoriales suaves de una variedad suave $M$, $\mathfrak{X}(M)$ será tanto un espacio vectorial como un módulo sobre el anillo de funciones.
