\section{El Espacio Cotangente}\label{Sección: Espacio Cotangente}
\subsection{El Espacio Dual de un Espacio Vectorial}

\begin{definition}[Espacio Dual]
	Sea $V$ un espacio vectorial sobre un campo $\mathbb{F}$. Definiremos el \it{espacio dual (algebraico) de $V$}, denotado como $V^*$, como el conjunto de todos los mapas (en el sentido algebraico) $\omega: V \to \mathbb{F}$. A un elemento del espacio dual, $\omega \in V^*$, se le llama \it{covector}.
\end{definition}

Hacemos la aclaración de que estaremos trabajando con el espacio dual algebraico y por conveniencia lo llamaremos simplemente el espacio dual o dual, en nuestro caso no hace ninguna diferencia ni hay perdida de generalidad dado que estamos trabajando con espacios vectoriales de dimensión finita, por lo que el dual topológico y el dual algebraico coincidirán.

\begin{example}
	Si $M$ es una variedad suave y $F: M \to \R$ es una función suave, el diferencial de $F$ en un punto $p \in M$, $dF: T_p(M) \to \R$ es un covector.
\end{example}

Como veremos a continuación, el espacio dual es un espacio vectorial en sí mismo bajo las operaciones de suma y producto escalar definidas del siguiente modo:
\begin{alignat*}{2}
	(\omega+\nu)(x) & =\omega(x)+\nu(x), \quad &  & \forall \omega,\nu \in V^{*},           \\
	(a\omega)(x)    & = a\omega(x), \quad      &  & \forall x\in V,\forall a \in\mathbb{F}.
\end{alignat*}
Más aún, este espacio vectorial es finito dimensional y la dimensión de $V^{*}$ coincide con la dimensión de $V$.

\begin{theorem}[Dimensión del Espacio Dual]\label{Teorema: Dimensión del Espacio Dual}
	Si $V$ es un espacio vectorial finito dimensional y $V^{*}$ es su espacio dual, entonces $\dim(V) = \dim(V^{*})$.
\end{theorem}

\begin{proof}
	Sea $\{E_1, \dots, E_n\}$ una base para $V$. Definamos $n$ covectores, $\{\epsilon_1, \dots, \epsilon_n\}$ en $V^{*}$ del siguiente modo:
	\[
		\epsilon_i(E_j) = \delta_{ij} = \begin{cases}
			1, & i = j    \\
			0, & i \neq j
		\end{cases},
	\]
	donde $\delta_{ij}$ es la delta de Kronecker, notemos que el conjunto $\{\epsilon_1, \dots, \epsilon_n\}$ es linealmente independiente. Por la linealidad de los covectores tenemos que si:
	\[
		\sum_{i=1}^{n} a_i \epsilon_i(x) = 0, \quad a_{i} \in \mathbb{F}, x \in V,
	\]
	entonces, si $x = E_j$:
	\begin{align*}
		\sum_{i=1}^n a_i \epsilon_i (E_j) & = \sum_{i=1}^n a_j \delta_{ij} = a_i = 0
	\end{align*}
	Esto implica que cada $a_i$ es nulo, por lo que $\{\epsilon_1, \dots, \epsilon_n\}$ es un conjunto linealmente independiente.

	Ahora mostraremos que cada elemento en $V^{*}$ puede ser representado como una combinación lineal de elementos de $\{\epsilon_1, \dots, \epsilon_n\}$. Cada $x \in V$ puede ser escrito como una combinación lineal:
	\[
		x = \sum_{i=1}^{n} \mu_{i} E_i, \quad \mu_i \in \mathbb{F}
	\]
	Por lo que si tomamos un covector $\epsilon \in V^{*}$ tendremos que para cada $x \in V$:
	\begin{align*}
		\epsilon(x) & = \epsilon
		\left (\sum_{i=1}^{n} \mu_i E_i \right)                          \\
		            & = \sum_{i=1}^{n} \mu_i \epsilon \left(E_i \right),
	\end{align*}
	por otro lado:
	\begin{align*}
		\epsilon_i(x) & = \epsilon_i\left(\sum_{i=1}^{n} \mu_i E_i\right) \\
		              & = \mu_i.
	\end{align*}
	Por lo tanto, cada $\epsilon$ puede ser representado como una combinación lineal en términos de $\epsilon_i$ de la siguiente forma:
	\[
		\epsilon = \sum_{i=1}^{n} \epsilon_i \epsilon(E_i).
	\]

	Esto demuestra que $\{\epsilon_1, \dots, \epsilon_n\}$ es una base para $V^{*}$ y, por lo tanto, $\dim(V) = \dim(V^{n})$.
\end{proof}

\begin{example}
	La base del espacio dual $V^{*}$ está dada por los vectores fila de la matriz inversa a la matriz formada por los vectores bases de $V$.

	Consideremos una base para $\R^n$ y denotemos dicha base por $\{E_1,\dots,E_n\}$, cada uno de estos vectores puede ser representado como una matriz $1 \times n$ del siguiente modo:
	\[
		E_1 = \begin{bmatrix}
			E_1^1  \\[12pt]
			E_1^2  \\[12pt]
			\vdots \\[12pt]
			E_1^n
		\end{bmatrix}, E_2 = \begin{bmatrix}
			E_2^1  \\[12pt]
			E_2^2  \\[12pt]
			\vdots \\[12pt]
			E_2^n
		\end{bmatrix}, \hdots, E_n = \begin{bmatrix}
			E_n^1  \\[12pt]
			E_n^2  \\[12pt]
			\vdots \\[12pt]
			E_n^n
		\end{bmatrix} \in V.
	\]
	Por el teorema anterior sabemos que la base del espacio dual puede ser obtenida definiendo $n$ covectores, $\{\epsilon_1,\dots,\epsilon_n\}$, del siguiente modo:
	\[
		\epsilon_i (E_j) = \delta_{ij}
	\]
	Este nos dará $n \times n$ ecuaciones, las cuales podemos representar con el siguiente sistema de ecuaciones:
	\[
		\begin{bmatrix}
			\epsilon_{1}^{1} & \epsilon_{1}^{2} & \hdots & \epsilon_{1}^{n} \\[12pt]
			\epsilon_{2}^{1} & \epsilon_{2}^{2} & \hdots & \epsilon_{2}^{n} \\[12pt]
			\vdots           & \vdots           & \ddots & \vdots           \\[12pt]
			\epsilon_{n}^{1} & \epsilon_{n}^{2} & \hdots & \epsilon_{n}^{n}
		\end{bmatrix}
		\begin{bmatrix}
			E_{1}^{1} & E_{2}^{1} & \hdots & E_{n}^{1} \\[12pt]
			E_{1}^{2} & E_{2}^{2} & \hdots & E_{n}^{2} \\[12pt]
			\vdots    & \vdots    & \ddots & \vdots    \\[12pt]
			E_{1}^{n} & E_{2}^{n} & \hdots & E_{n}^{n}
		\end{bmatrix} = \begin{bmatrix}
			1      & 0      & \hdots & 0      \\
			0      & 1      & \hdots & 0      \\
			\vdots & \vdots & \ddots & \vdots \\
			0      & 0      & \hdots & 1
		\end{bmatrix}.
	\]

	Llamaremos a la matriz formada por los coeficientes de los covectores $A$ y a la matriz formada por los coeficientes de la base de $V$, $B$. Claramente $B$ es una matriz dado que está formada por vectores de la base, que linealmente independientes. Por lo que para $A$ se tendrá la ecuación:
	\[
		A = IB^{-1} = B^{-1}.
	\]
	Por lo tanto, $A$ estará formado por los vectores filas de $B^{-1}$.

	En particular, si consideramos la base estándar de $\R^n$ esto nos dice que una base, a la cual llamaremos la base estándar del espacio dual, será:
	\begin{align*}
		\epsilon_1 & = \begin{bmatrix} 1 & 0 & \cdots & 0\end{bmatrix}, \\
		\epsilon_2 & = \begin{bmatrix} 0 & 1 & \cdots & 0\end{bmatrix}, \\
		           & \vdots                                             \\
		\epsilon_n & = \begin{bmatrix} 0 & 0 & \cdots & 1\end{bmatrix}.
	\end{align*}
	Esto tiene mucho sentido si pensamos en los vectores columna como matrices $n \times 1$, esto es, transformaciones lineales de $\mathbb{F}$ a $\mathbb{F}^n$, y los covectores son transformaciones lineales de $\mathbb{F}^n$ a $\mathbb{F}$, matrices $1 \times n$
\end{example}

\begin{definition}[Mapa Dual]
	Sean $V$ y $W$ espacios vectoriales y sea $A: V \to W$ un mapa lineal. Definiremos al mapa $A^*: W^* \to V^*$ como:
	\[(A^{*}\omega)(v) = \omega(Av), \quad \omega \in W^{*}, v \in V.\]
	Llamaremos a este mapa, el \it{mapa dual} o \it{transpuesta de $A$}.
\end{definition}

\begin{lemma}
	El mapa $A^{*}\omega$ es un covector en $V$ y $A^{*}$ es un mapa lineal.
\end{lemma}

\begin{proof}
	Para ver que $A^{*}\omega$ es un covector basta notar que $A: V \to W$ y $\omega: W \to \mathbb{F}$, por lo que la composición $\omega \circ A: V \to \mathcal{F}$ es un mapa que va de $V$ al campo $\mathbb{F}$, lo cual es, por definición, un covector en $V$.

	La linealidad de $A^{*}$ se tiene del hecho de que la composición de mapas lineal es un mapa lineal.
\end{proof}

\begin{lemma}
	Los mapas duales satisfacen las siguientes propiedades:
	\begin{itemize}
		\item $(A \circ B)^{*} = B^{*} \circ A^{*}$.
		\item $(\id_V)^{*} = V^{*} \to V^{*}$ es el mapa identidad en $V$.
	\end{itemize}
\end{lemma}

\begin{proof}
	Sean $V,W$ y $X$ espacios vectoriales sobre un campo $\mathcal{F}$, $A: W \to V$ y $B: X \to W$ mapas lineales, para cada $\omega \in V^{*}$:
	\[ (A\circ B)^{*}(\omega) = \omega(A \circ B). \]
	Por otro lado:
	\begin{align*}
		(B^* \circ A^*)(\omega) & = B^{*}(A^*(\omega))        \\
		                        & = B^{*}(\omega(A))          \\
		                        & = \omega(A) \circ B         \\
		                        & = (\omega \circ A) \circ B.
	\end{align*}
	Para la identidad, $\id_V: V \to V$, para cada $\omega \in V^{*}$ y cada $v \in V$ tenemos:
	\begin{align*}
		(\id_V)^{*}(\omega)(v) & = \omega(\id_V(v)) \\
		                       & = \omega(v)        \\
		                       & = \id_{V^{*}}.
	\end{align*}
\end{proof}

Al ser $V^{*}$ un espacio vectorial en sí mismo, es posible definir su dual, al cual llamaremos \it{segundo espacio dual} o \it{espacio bidual}, y que denotaremos por $V^{**}$, a los elementos del espacio bidual los denotaremos con letras griegas mayúsculas, esto simplemente para distinguirlos de los elementos el espacio dual.

El bidual es también un espacio con propiedades bastante interesantes, sus elementos son funcionales lineales que toman covectores del espacio dual y regresan escalares, pero quizá la propiedad más interesante es que existe un isomorfismo natural entre el bidual $V^{**}$ y $V$, el cual no depende de la elección de bases.

\begin{theorem}[Teorema de Reflexividad]
	Si $V$ es un espacio vectorial y $V^{**}$ es su espacio bidual, $V$ es naturalmente isomorfo a $V^{**}$.
\end{theorem}

\begin{proof}
	Tomemos el operador $\Phi: V \to V^{**}$ como sigue, para cada $v \in V$ definimos el funcional lineal $\Phi(v): V^{*} \to \mathbb{F}$.
	\[
		\Phi(v)(\omega) = \omega(v), \quad v \in V, \omega \in V^{*}
	\]
	Primero mostraremos que para cada $v \in V$, $\Phi(v)(\omega)$ depende de forma lineal de $\omega$, esto implicará que $\Phi(v) \in V^{**}$ dado que $\omega(v) \in \mathbb{F}$. Sean $v \in V$, $\omega,\nu \in V^{*}$ y $c \in \mathbb{F}$, tendremos que:
	\begin{align*}
		\Phi(v)(\omega + \nu) & = (\omega + \nu)(v)              \\
		                      & = \omega(v) + \nu(v)             \\
		                      & = \Phi(v)(\omega) + \Phi(v)(\nu) \\
		\Phi(v)(c \omega)     & = (c\omega)(v)                   \\
    &= c(\omega(v))\\
    &= c(\Psi(v)(\omega))
	\end{align*}

  Ahora mostraremos que $\Phi: V \to V^{**}$ es un funcional lineal. Sean $u,v \in V$, $\omega \in V^{*}$ y $c \in \mathbb{F}$, tenemos que:
	\begin{align*}
		\Phi(u + v)(\omega) & = \omega(u + v)                     \\
		                    & = \omega(u) + \omega(v)             \\
		                    & = \Phi(u)(\omega) + \Phi(v)(\omega) \\
		\Phi(c v)(\omega)   & = \omega(cv)                        \\
		                    & = c\omega(v)                        \\
		                    & = c \Phi(v)(\omega)
	\end{align*}

	Ahora, por el teorema \ref{Teorema: Dimensión del Espacio Dual} se garantiza que $\dim(V) = \dim(V^{*}) = \dim(V^{**})$, por lo que solamente necesitaremos mostrar que $\Phi$ es un funcional inyectivo.

	Consideremos $v \in V$ tal que $\Phi(v)(\omega) = 0$ para cada $\omega \in V^{*}$, esto por definición implica que $\omega(v) = 0$ para cada $\omega$, y el único elemento en $V$ con dicha propiedad es el elemento nulo, por lo tanto $\Phi$ es inyectivo, con lo cual concluimos que es un isomorfismo entre $V$ y $V^{**}$.
\end{proof}
