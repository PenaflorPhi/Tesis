\section{Campos Vectoriales}\label{Sección: Campos Vectoriales}
\subsection{Fibrados Vectoriales}\label{Subsección: Fibrados Vectoriales}
Siguiendo con la extensión y generalización que hemos estado realizando de conceptos conocidos de cálculo en espacios euclidianos, ahora extenderemos la idea de los campos vectoriales, estos objetos nos darán una forma de asignar a cada punto de una variedad un vector del espacio tangente.

Para poder definir lo que es un campo vectorial primero hablaremos de lo que son los fibrados vectoriales, lo cual responderá a las preguntas ¿Por qué hemos decido llamar fibrados a la colección de espacios tangentes a una variedad?, y ¿qué es una fibra?

\begin{definition}[Fibrado Vectorial]
	Sean $E$ y $M$ variedades suaves, $\pi: E \to M$ un mapa suave y sobreyectivo, al cual llamaremos la \it{proyección fibrada}, diremos que $\pi$ es \it{localmente trivial de rango $k$} si:

	\begin{itemize}
		\item Para cada $p \in M$, su preimagen, $\pi^{-1}(\{p\})$, tiene estructura de espacio vectorial con dimensión $k$, llamaremos a esta preimagen la \it{fibra en $p$} y usualmente se denotará como $E_p$.

		\item Para cada $p \in M$ existe una vecindad abierta $U$ que lo contiene y $\phi: \pi^{-1}(U) \to U \times \R^{k}$, donde $\phi$ es un difeomorfismo tal que para cada $q \in U$ la restricción:
      \[ (\phi|_{\pi^{-1}(\{q\})}): \pi^{-1}(\{q\}) \to \{q\} \times \R^{k} \]
		      es un isomorfismo de espacios vectoriales.
	\end{itemize}

	Diremos que $U$ es un \it{conjunto abierto trivializante} en $E$. La colección $\{(U_\alpha, \phi_\alpha)\}_{\alpha}$, donde $\{U_{\alpha}\}_{\alpha}$ es una cubierta abierta de $M$ es llamada la trivialización local de $E$, y la cubierta abierta $\{U_\alpha\}_{\alpha}$ es llamada la cubierta abierta trivializante de $M$ para $E$.

  Un \it{fibrado vectorial suave de rango $k$} es una terna $(E,M,\pi)$ donde $E$ y $M$ son variedades suaves y $\pi: E \to M$ es un mapa suave y sobreyectivo que es localmente trivial de rango $k$. A la variedad $E$ la llamamos el \it{espacio total} del fibrado vectorial y a la variedad $M$ el \it{espacio base}. Por simplicidad se dice que \it{$E$ es un fibrado vectorial sobre $M$}.
\end{definition}

\begin{example}[Fibrados Tangentes]
	Por los resultados mostrados en la sección anterior es evidente que si $M$ es una variedad suave, la terna $(TM, M, \pi)$, donde $TM$ es el fibrado tangente y $\pi$ es la proyección natural de $TM$ sobre $M$, será un fibrado vectorial suave.
\end{example}

\begin{example}[Producto Fibrado]
	Dada una variedad $M$, sea $\pi: M \times \R^{k} \to M$ la proyección sobre el primer término. Entonces $(M \times \R^{k}, M, \pi)$ es un fibrado tangente de rango $k$ llamado el \it{producto fibrado} de rango $k$ sobre $M$. La estructura de espacio vectorial en la fibra $\pi^{-1}(p) = \{(p,v) | v \in \R^{k}\}$ es:
	\[
		(p,u) + (p,v) = (p, u+v), \quad c(p,v) = (p,cv), \quad c \in \R.
	\]
	Una trivialización local en $M \times \R$ esta dada por la identidad $\id_{M \times \R}: M \times \R \to  M \times \R$.
\end{example}

\begin{definition}[Sección de un Fibrado Vectorial]
	Una \it{sección} de un fibrado vectorial $\pi: E \to M$ es un mapa $\sigma: M \to E$ tal que $\pi \circ \sigma = \id_{M}$, esto quiere decir que para cada $p \in M$, $\sigma(p)$ pertenece a $E_p$.

	Diremos que el mapa $\sigma$ es una \it{sección suave} si $\sigma$ es suave de $M$ a $E$. Denotaremos al conjunto de todas las secciones suaves de $E$ como $\Gamma(E)$.
\end{definition}

Como se vio en la sección \ref{Sección: Mapas Suaves}, el conjunto de funciones suaves, $C^{\infty}(M)$, es un anillo conmutativo con identidad bajo las operaciones de suma y producto, una de las propiedad más importantes de los fibrados vectoriales es que estos son módulos sobre el anillo de funciones suaves.

\begin{theorem}\label{Teorema: Los Fibrados Vectoriales Son Modulos}
	Sea $(E,M, \pi)$ un fibrado vectorial suave, $\sigma, \tau \in \Gamma(E)$ y sea $f \in C^{\infty}(M)$. Si definimos la suma de secciones suaves y el producto por una función como sigue:
	\begin{align*}
		(\sigma + \tau)(p) & = \sigma(p) + \tau(p), \quad p \in M. \\
		(f\sigma)(p)       & = f(p)\sigma(p), \quad p \in M.
	\end{align*}
	Entonces $\Gamma(E)$ es un módulo sobre el anillo $C^{\infty}(M)$.
\end{theorem}

\begin{proof}
	Dado que tanto $\sigma$ como $\tau$ son secciones tendremos que elegido un punto $p \in M$, $\sigma(p)$ y $\tau(p)$ pertenecen a la fibra $E_p$, y como $E_p$ tiene estructura de espacio vectorial se tiene que $(\sigma + \tau)(p) \in E_p$, por lo tanto, $\sigma + \tau$ es una sección del fibrado. Para mostrar la suavidad tomamos un punto $p \in M$ y un conjunto abierto trivializante en $E$ que contenga a $p$ con una trivialización suave
	\[
		\phi: \pi^{-1}(U) \to U \times \R^k.
	\]
	Supongamos que para $q \in M$ se tiene:
	\begin{align*}
		\phi \circ \sigma(q) & = (q, s_1(q), \dots, s_k(q)), \\
		\phi \circ \tau(q)   & = (q, t_1(q), \dots, t_k(q)).
	\end{align*}

	Como $\sigma$ y $\tau$ son mapas suaves, $\{s_i\}_{i=1}^k$ y  $\{t_i\}_{i=1}^k$ serán funciones suaves, y por definición $\phi$ es isomorfismo lineal, por lo que
	\[
		\phi \circ (\sigma + \tau)(q) = (q, (s_1 + t_1)(q), \dots, (s_k + t_k)(q)).
	\]

	Por lo tanto, $\sigma + \tau$ es un mapa suave en cada punto de $U$, en particular será suave en $p$. Se concluye que $\sigma + \tau$ es una sección suave.

	De modo similar, $f\sigma$ será una sección del fibrado vectorial dado que al elegir un punto $p \in M$ se tiene que por definición del producto que $(f \circ \sigma)(p) = f(p)\sigma(p)$ y como $E_p$ tiene estructura de espacio vectorial y $f(p)$ es una simple constante $f(p)\sigma(p)$ pertenecerá a la fibra $E_p$.

	Ahora tomemos un punto $p \in M$ y un conjunto $U$ trivializante en $E$ que contenga a $p$ con trivialización suave
	\[
		\phi: \pi^{-1}(U) \to U \times \R^{k}.
	\]
	Supongamos que para $q \in M$ se tiene:
	\[
		\phi \circ \sigma(q) = (q, s_1(q), \dots, s_k(q)).
	\]

	Como se ha mencionado anteriormente, al ser $\sigma(q)$ un mapa suave, cada $\{s_i\}_{i=1}^k$ será una función suave en $M$. Por la linealidad de $\phi$ se tendrá que
	\[
		\phi \circ (f \sigma)(q) = (q, fs_1(q),\dots, fs_k(q)).
	\]

	Por lo tanto, cada una de las componentes es suave, así se garantiza que el producto definido de esta manera es una sección suave. Más aún, $\Gamma(E)$ será tanto un espacio vectorial como un módulo sobre el anillo $C^{\infty}(M)$.
\end{proof}

Los módulos, al ser una generalización del concepto de espacio vectorial también nos permitirán generalizar varios otros conceptos relacionados, como lo son las combinaciones lineales, la dependencia (o independencia) lineal o las bases, sin embargo, es importante no perder de vista que no todos los resultados que aplican para espacios vectoriales aplicaran para los módulos.
