\subsection{El Espacio Cotangente}\label{Subsección: El Espacio Cotangente}
Ahora que sabemos qué es el espacio dual podemos hablar del espacio dual al espacio tangente a un punto de una variedad, espacio al cual llamaremos \it{espacio cotangente}.

\begin{definition}[Covector en un punto]
	Sea $M$ una variedad suave y $p$ un punto en $M$. Definimos el \it{espacio cotangente de $M$ en $p$}, denotado $T_{p}^{*}(M)$, como el espacio dual del espacio tangente $T_p(M)$
	\[
		T^{*}_p(M) = (T_p(M))^*.
	\]
	Llamaremos \textit{covector tangente a $p$}, o \textit{covector en $p$} a los elementos de $T_p^{*}(M)$.
\end{definition}

En la subsección \ref{Subsección: Espacios Tangentes en Coordenadas} estudiamos cómo es que, dado un punto $p$ en una variedad $M$ suave y una carta suave $(U,\phi) = (U,\phi_1,\ldots,\phi_n)$ que lo contenga, podemos escribir a cualquier vector en el espacio tangente a $M$ en $p$ como una combinación lineal. Retomando la notación vista en la sección de Campos Vectoriales \ref{Sección: Campos Vectoriales}, si $X(p)$ es un vector en $T_p(M)$, entonces podemos expresarlo como la combinación lineal:

\[
	X(p) = \sum_{i=1}^{n} X_i(p) \left. \frac{\partial}{\partial \phi_i} \right|_{p}.
\]

Dado que estamos definiendo al espacio cotangente $T_{p}^{*}(M)$ como el espacio dual de $T_p(M)$, es posible expresar a cualquier vector en $\omega \in T_{p}^{*}(M)$ como una combinación lineal, y más aún, la base que utilizamos para representar a los vectores del espacio cotangente puede estar asociada a una base para el espacio tangente. Denotaremos a la base de $T_{p}^{*}(M)$ por $\{\lambda_1|_p, \ldots, \lambda_n|_p\}$, más adelante veremos qué son estos elementos de la base y les daremos una notación diferente, por ahora bastará saber que si $\omega \in T_{p}^{*}(M)$, entonces lo podemos escribir como la combinación lineal:
\[
	\omega = \sum_{i=1}^{n} \omega_i \left. \lambda_i \right|_p,
\]
donde $\omega_i = \omega \left( \left. \frac{\partial}{\partial \phi_i} \right|_p\right)$.

Al estar expresando a los vectores cotangentes como combinaciones lineales estamos trabajando con coordenadas, por lo que estas combinaciones lineales dependerán de nuestra elección de coordenadas. En la subsección \ref{Subsección: Espacios Tangentes en Coordenadas} vimos cómo es que podemos transformar de un sistema coordenado a otro.

Dado un subconjunto abierto $U$ en una variedad suave $M$ podemos dar diferentes coordenadas locales en $U$, digamos $\{\phi_1,\ldots,\phi_n\}$ y $\{\psi_1,\ldots,\psi_n\}$. Hemos demostrado anteriormente que parea transformar de un sistema coordenado a otro es posible utiliza la igualdad:
\[
	\left. \frac{\partial}{\partial \phi_i} \right|_p =
	\sum_{j=1}^{n} \left. \frac{\partial \psi_j}{\partial \phi_i} (\phi(p))
	\frac{\partial}{\partial \psi_j} \right|_{p}.
\]
Con esta igualdad y considerando un vector $\omega \in T_{p}^{*}(M)$ procederemos del siguiente modo, si tenemos dos bases $\{\lambda_1,\ldots,\lambda_n\}$ y $\{\hat{\lambda_1},\ldots,\hat{\lambda_n}\}$, las cuales estén asociadas a las bases $\{\phi_1, \ldots, \phi_n\}$ y $\{\psi_1, \ldots, \psi_n\}$ respectivamente, entonces podemos expresar a $\omega$ como:
\[
	\omega = \sum_{i=1}^{n} \left. \omega_i \lambda_i \right|_p
	= \sum_{j=1}^{n} \left. \hat{\omega}_j \hat{\lambda}_j \right|_p.
\]
Para transformar de un sistema coordenado al otro utilizaremos la igualdad anterior y la manera que hemos dado para calcular a los coeficientes $\omega_i$, obteniendo la siguiente identidad.
\begin{align*}
	\omega_i & = \omega
	\left(
	\left. \frac{\partial}{\partial \phi_i} \right|_{p}
	\right)                                                                       \\
	         & = \omega
	\left(
	\sum_{j=1}^{n} \frac{\partial \psi_j}{\partial \phi_i} (\phi(p))
	\left. \frac{\partial}{\partial \psi_j}\right|_{p} \right)                    \\
	         & = \sum_{j=1}^{n} \frac{\partial \psi_j}{\partial \phi_i} (\phi(p))
	\omega \left( \left. \frac{\partial}{\partial \psi_j}\right|_{p} \right)      \\
	         & = \sum_{j=1}^{n} \frac{\partial \psi_j}{\partial \phi_i} (\phi(p))
	\hat{\omega_j}
\end{align*}

Los espacios cotangentes comparten muchas propiedades con los espacios tangentes, ambos son espacios vectoriales definidos para cada punto en una variedad suave. Al ser este el caso también es posible estudiar qué es lo que sucede al considerar todos (o un subconjunto) de los espacios cotangentes a una variedad de manera simultánea. Cuando trabajamos con los espacios tangentes definimos el fibrado tangente, como es de esperarse, ahora definiremos el \it{fibrado cotangente} y describiremos algunas de sus propiedades.

\begin{definition}
	Sea $M$ una variedad suave, llamaremos a la unión disjunta de todos los espacios cotangentes, la cual denotaremos por $T^{*}M$, el \it{fibrado cotangente}.
	\[ T^{*}M = \bigsqcup_{p \in M} T_{p}^{*}(M) = \bigcup_{p \in M} \{p\} \times T_{p}^{*}(M). \]
\end{definition}

Para el fibrado cotangente también existe un mapa $\pi: T^{*}M \to M$ que nos induce una topología en $T^{*}M$ la cual hace de $T^{*}M$ una variedad topológica, a la que es posible darle una estructura suave; como cabría esperarse, este mapa está dado como $\pi(\omega) = p$, donde $\omega \in T_p^{*}(M)$.

Como acabamos de mencionar, dada una carta suave $(U,\phi) = (U,\phi_1,\ldots,\phi_n)$ en $M$, denotaremos por $\{\lambda_1|_p,\ldots,\lambda_n|_p\}$ a la base de $T^*(M)$, esta base estará asociada, en el sentido de ser dual, a la base $\{\frac{\partial}{\partial \phi_1}|_p, \ldots, \frac{\partial}{\partial \phi_n}|_p\}$. Así, estamos definiendo $n$ mapas $\lambda_i: U \to T^{*}M$, a estos mapas les llamaremos \it{campos de covectores coordenados}.

\begin{theorem}[El Fibrado Cotangente como Fibrado Vectorial]
	Sea $M$ una variedad suave de dimensión $n$. El fibrado cotangente $T^{*}M$, con su mapa de proyección y su estructura de espacio vectorial en cada fibra, tiene una topología y una estructura suave que lo hacen un fibrado vectorial de rango $n$ sobre $M$.
\end{theorem}

\begin{proof}
	Sea $(U,\phi) = (U,\phi_1, \ldots, \phi_n)$ una carta suave en $M$ y $p$ un punto en $U$. Cada $\omega \in T_{p}^{*}(M)$ puede ser representado de manera única como una combinación lineal:
	\[
		\omega = \sum_{i=1}^{n} \omega_i \lambda_i |_p,
	\]
	donde $\{\lambda_1, \ldots, \lambda_n \}$ es la base dual a $\{\frac{\partial}{\partial \phi_1}, \ldots, \frac{\partial}{\partial \phi_n}\}$. La unicidad de esta representación nos da lugar para poder definir el siguiente mapa, el cual es trivialmente una biyección.
	\begin{align*}
		\Phi: T^*U                                   & \to U \times \R^n \\
		\Phi( \sum_{i=1}^{n} \omega_i \lambda_i |_p) & =
		(p, \omega_1, \ldots, \omega_n).
	\end{align*}
	Ahora, si $\mathcal{A}$ es la estructura suave maximal para $M$ y consideramos el dominio $U$ de cualquier carta suave podemos definir el conjunto $\mathcal{B}_U$ conformado por la colección de todos los subconjuntos abiertos de $T^{*}U$, y, con base en estos conjuntos definiremos al conjunto $\mathcal{B}$, el cual será la unión de todos los $\mathcal{B}_U$.

	El conjunto $\mathcal{B}$ es nuestro candidato natural para ser la base de la topología con la que dotaremos a $T^{*}M$, basta mostrar que, en efecto, $\mathcal{B}$ cumple el ser la base para una topología.

	Es evidente que si $\mathcal{A}$ es la estructura suave maximal en $M$, entonces:
	\[
    T^{*}M = \bigsqcup_{a \in \mathcal{A}} T^{*}U_{\alpha}
    \subset \bigcup_{U \in \mathcal{B}} U = T^*M.
	\]
	Luego, si $U$ y $V$ son los dominios de dos cartas suaves en $M$ y $A$ y $B$ son subconjuntos abiertos de $T^{*}U$ y $T^{*}V$ respectivamente entonces, $A \cap B$ es un subconjunto abierto de $T^{*}(U \cap V)$, esto dado que $A \cap T^{*}(U \cap V)$ es abierto en $T^{*}(U \cap V)$, y de modo similar $B \cap T^{*}(U \cap V)$ es abierto en $T^{*}(U \cap V)$, por lo que la siguiente cadena de igualdades será verdadera:
	\begin{align*}
		A \cap B & = A \cap B \cap T^{*}U \cap T^{*}V                       \\
		         & = A \cap B \cap T^{*}(U \cap V)                          \\
		         & = (A \cap T^{*}(U \cap V)) \cap (B \cap T^{*}(U \cap V)).
	\end{align*}
  Por lo tanto, $A \cap B$ es abierto en $T^{*}(U \cap V)$, y así se concluye que $\mathcal{B}$ es una subbase para la topología en $T^{*}M$.

  La estructura suave de $T^{*}M$ se obtiene fácilmente al considerar los mapas $\Phi = (\phi_1 \circ \pi, \ldots, \phi_n \circ \pi, \omega_1, \ldots, \omega_n)$ como los mapas coordenados para las cartas suaves en $T^{*}M$, la demostración de este hecho es idéntica a la dada para mostrar que $TM$ también es una variedad suave.

  Sabiendo todo esto, la terna $(T^{*}M, M, \pi)$ es, por definición, un fibrado vectorial de rango $n$ sobre M.
\end{proof}

Otra similitud entre el fibrado cotangente $T^{*}M$ y el fibrado tangente $TM$ de una variedad suave es que, dadas coordenadas locales suaves $\{\phi_1,\ldots,\phi_n\}$ para un subconjunto abierto $U$ de $M$, estas nos permiten dar coordenadas locales suaves en el fibrado cotangente. La carta $(\pi^{-1}(U), \hat{\phi})$, donde $\hat{\phi}: \pi^{-1}(U) \to \phi(U) \times \R^n$ es el mapa definido como:
\[
  \hat{\phi} \left(\sum_{i=1}^{n} \omega_i \lambda_i |p \right) 
  = \left(\phi_1(p), \ldots, \phi_n(p), \omega_1(p), \ldots, \omega_n(p) \right),
\]
es una carta coordenada suave en $T^{*}M$, esto dado que $\hat{\phi}$ se puede expresar como la composición de los mapas $(\phi \times \id_{\R^n}) \circ \Phi$, donde $\Phi$ es la trivialización asociada a $(U,\phi)$ y que cada uno de los mapas en la composición es un difeomorfismo. Llamamos a estas coordenadas, las \it{coordenadas naturales para $T^{*}M$}.
