\subsection{El Espacio Cotangente}\label{Subsección: El Espacio Cotangente}
Ahora que sabemos que es el espacio dual podemos hablar del espacio espacio dual del espacio tangente a un punto de una variedad, espacio al cual llamaremos \it{espacio cotangente}.

\begin{definition}
  Sea $M$ una variedad suave y $p$ un punto en $M$. Definimos el \it{espacio cotangente de $M$ en $p$}, denotado $T_{p}^{*}(M)$, como el espacio dual del espacio tangente $T_p(M)$
\[
  T^{*}_p(M) = (T_p(M))^*.
\]
  Llamaremos \textit{covector tangente a $p$}, o \textit{covector en $p$} a los elementos de $T_p^{*}(M)$.
\end{definition}

Como se vio en el teorema \ref{Teorema: Dimensión del Espacio Dual} para cada base $\left\{\frac{\partial}{\partial \phi_1}, \ldots, \frac{\partial}{\partial \phi_n}\right\}$ es posible construir una base para el espacio dual $T_p^*(M)$, la cual denotaremos como $\{\lambda_1, \ldots, \lambda_n\}$ para la cual se cumpla:
\[
  \lambda_i \left(\frac{\partial}{\partial \phi_j}\right) = \delta_{ij},
\]
donde $\delta_{ij}$ es la delta de Kronecker. 
