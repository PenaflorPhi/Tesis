\subsection{Aplicaciones Tangentes de Campos Vectoriales}\label{Subsección: Pushforward de Campos Vectoriales}

El diferencial de un mapa suave en punto nos da la mejor aproximación lineal en el espacio tangente para una vecindad de dicho punto, es posible extender está idea a los campos vectoriales.

\begin{definition}[Pushforward]
	Sea $F: M \to N$ un mapa suave entre variedades, y sea $d_pF: T_p(M) \to T_{F(p)}(N)$ el diferencial de $F$ en el punto $p$. Si $X(p)$ es un vector tangente en $T_p(M)$ diremos que $d_{p}F(X(p))$ es el \it{pushforward} o la \it{aplicación tangente} de $X(p)$.
\end{definition}

El pushforward en general no es un campo vectorial, esto se observa fácilmente si consideramos un campo vectorial $X$ y dos punto $p,q \in M$ para los cuales se tenga que $t = F(p) = F(q)$, $X(p)$ y $X(q)$ serán aplicaciones tangentes en un mismo punto de $M$, sin embargo, esto no implica que $d_t(X(p)) = d_t(X(q))$.

Más en general si $X$ es un campo vectorial y $F$ no es inyectiva existirán punto de $N$ para los cuales podemos obtener diferentes campos vectoriales al aplicarle al tomar el pushforward $dF(X)$ a diferentes puntos, y si $F$ no es sobreyectiva entonces no es posible asignar un vector del espacio tangente a ningún punto en $q$ que pertenezca a $N - F(M)$.

Para que el resultado de nuestro pushforward sea siempre un campo vectorial es necesario imponer una condición, que $F$ sea un difeomorfismo, para demostrar esto primero daremos la siguiente definición:

\begin{definition}[Campos Vectoriales Relacionados]
	Sea $F: M \to N$ un mapa suave entre variedades y $X$ un campo vectorial en $M$. Diremos que el campo vectorial $X$ está \it{relacionado por el mapa $F$} a un campo vectorial $Y$ en $N$ si para cada punto $p$ de $M$ se tiene que
	\[
		d_pF(X(p)) = Y_{F(p)}.
	\]
\end{definition}

\begin{lemma}
	Supongamos que $F: M \to N$ es un mapa suave entre variedades, $X$ es un campo vectorial suave en $M$ y $Y$ es un campo vectorial suave en $N$. Entonces $X$ e $Y$ están relacionados por $F$ si y solo si para cada función suave real valuada $f$ que este definida en una vecindad de $F(p)$; se tiene que:
	\[
		X(f \circ F) = (Yf) \circ F.
	\]
\end{lemma}

\begin{proof}
	Sea $p$ un punto de $M$ y sea $f: V \subseteq N \to \R$ una función suave, donde $V$ es una vecindad de $F(p)$. Por un lado tenemos que:
	\begin{align*}
		X(f \circ F)(p) & = X(p)(f \circ F) \\
		                & = d_pF(X(p))f,
	\end{align*}
	Por el otro lado tendremos:
	\begin{align*}
		(Yf) \circ F(p) & = (Yf)(F(p)) \\
		                & = Y(F(p))f,
	\end{align*}
	De estas dos igualdades tendremos que $X(f \circ F) = (Yf)\circ F$ si $d_pF(X(p)) = Y(F(p))$ para cada $p \in M$, lo cual ocurre por definición si y solo si $X$ e $Y$ son campos relacionados por el mapa $F$.
\end{proof}

\begin{theorem}
	Sean $M$ y $N$ variedades suaves y sea $F: M \to N$ un difeomorfismo. Para cada campo vectorial $X \in \mathfrak{X}(M)$ existe un único campo vectorial suave $Y \in \mathfrak{X}(N)$ para el cual $X$ e $Y$ están relacionados por $F$.
\end{theorem}

\begin{proof}
	Por el resultado anterior sabemos que dos campos vectoriales $X$ e $Y$ están relacionados si $d_p F(X(p))= Y(F(p))$ para cada punto $p$ en $M$. Sabiendo esto podemos definir el campo $Y$ que queremos para cada $q \in N$ como:
	\[
		Y(q) = d_{F^{-1}(q)} F ( X (F^{-1}(q))).
	\]
	Dado que $F$ es un difeomorfismo $Y(q)$ es único lo cual garantiza la unicidad de este campo, además, por estar definido como la composición de mapas suaves $Y$ es un campo suave.
\end{proof}

\begin{definition}[Pushforward De Un Campo]
  Diremos que $dF(X)$ es el \it{pushforward del campo $X$ por $F$} si $F$ es un difeomorfismo, en cuyo caso $dF(X)$ es el campo suave definido de manera única visto en el teorema anterior.
\end{definition}

Podemos interpretar al pushforward de un campo vectorial no solo como una generalización de la derivada total, del mismo modo que como se vio en la sección \ref{Subsección: Espacios Tangentes en Coordenadas}, también podemos imaginarlo como una función que está empujando (\it{pushing}) a los vectores de una sección del fibrado tangente de una variedad $M$ a una sección del fibrado de otra variedad $N$; bajo está interpretación y como se estudiará más adelante, es posible moverlos hacía adelante (\it{pushing}) y hacía atrás (\it{pulling}), esto con ayuda del $\it{pull-back}$, objeto que estudiaremos en la siguiente sección.
