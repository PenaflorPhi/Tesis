\chapter{Variedades Riemannianas}\label{Capitulo: Variedades Riemannianas}

Nuestro objetivo principal es poder describir las curvas geodésicas en el disco de Poincaré, hasta ahora ahora hemos hablado de lo que son las variedades topológicas y como se les puede dar una estructura suave, hemos trasladado conceptos ya conocidos del cálculo en espacios euclidiano a las variedades; esto tiene ha tenido una finalidad, y es que, en la primera sección de esta tesis lo que realmente hemos hecho es responder a la pregunta \enquote{¿Dónde estamos midiendo?} y en la segunda sección hemos respondido a la pregunta \enquote{¿Qué herramientas podemos utilizar para medir?}. 

En esta sección nuestro objetivo es desarrollar, utilizando los conocimientos adquiridos en las secciones anteriores, las herramientas que finalmente nos permitirán describir a las curvas geodésicas, no solo del disco de Poincaré, sino también de otras variedades.

\section{El tensor de Riemann}
