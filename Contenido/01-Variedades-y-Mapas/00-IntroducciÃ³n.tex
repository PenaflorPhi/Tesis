\chapter{Introducción}
La geometría diferencial tiene como objetivo el estudio de las propiedades geométricas de las curvas y superficies a través del uso del cálculo diferencial e integral. Esta rama de las matemáticas tiene formalmente sus orígenes en siglo XIX, con las investigaciones realizadas por matemáticos como Carl Friedrich Gauss, Bernhard Riemann y Nikolái Lobachevsky sobre las propiedades de las superficies con curvatura. 

Si bien en sus inicios la geometría diferencial fue estudiada desde un punto de vista extrínseco, considerando a las curvas y superficies como partes de algún espacio ambiente más grande, heredando de manera natural propiedades de estos espacios, resultados como el \textit{theorema egregium} de Gauss resaltaron la importancia de las propiedades intrínsecas a las curvas y superficies, como lo son la longitud de una curva o la curvatura de una superficie.

Desde entonces ha habido grandes avances en esta área, los trabajos realizados durante el siglo XX por Henri Poincaré y Felix Hausdorff sobre los fundamentos formales de la topología dieron lugar a que matemáticos como Élie Cartan y Hassler Whitney replantaran la geometría diferencial en términos de lo que ahora conocemos como \textit{teoría de variedades}, la cual nos ha permitido generalizar las nociones de curva y superficies a dimensiones arbitrarias, librándonos además de la necesidad de tener que considerar a estos objetos como parte de un espacio ambiente más grande.

La teoría de variedades y la geometría diferencial han probado ser herramientas indispensables, no solo para las matemáticas, siendo de gran utilidad en el estudio de la topología, el análisis complejo, la geometría algebraica; sino también para la física, dando el marco matemático bajo el cual se entienden teorías como la relatividad general de Einstein o la teoría de gauge.

El presente trabajo está divido en tres capítulos.

El primer capítulo, titulado \textit{Variedades y mapas}, tiene como objetivo presentar definiciones y ejemplos básicos que permitan familiarizarnos con los objetos con los que estamos trabajando, las variedades, los cuales son espacios topológicos que localmente se asemejan a $\mathbb{R}^{n}$; así como dotar a las variedades de una estructura adicional a la cual llamamos \textit{atlas}, dicha estructura nos da una manera bien definida de lo que significa que una función sea suave.

El segundo capítulo, al cual llamamos \textit{Cálculo en variedades}, tiene por objetivo trasladar conceptos conocidos del cálculo multivariable a las variedades.

Finalmente, en el tercer capítulo, \textit{Geometría Riemanniana}, utilizamos las herramientas desarrolladas en el segundo capítulo para dotar de propiedades geométricas a las variedades, concluyendo con el cálculo de algunas geodésicas en distintas variedades.
