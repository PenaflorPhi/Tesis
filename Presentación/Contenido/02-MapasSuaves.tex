\begin{frame}
\begin{definition}[Función Suave \textbf{Incorrecta}]
  Sea $M$ una variedad topológica $n-$dimensional, $k$ un entero positivo y $f: M \to \R^{k}$ una función cualquiera, diremos que la función $f$ es suave si para cada carta $(U,\phi)$ se tiene que la composición $f \circ \phi^{-1}: \phi(U) \subseteq \R^n \to \R^k$ es diferenciable
\end{definition} \pause

\vspace{12pt}
\centering
Consideremos los siguientes atlas suaves en $\R$: 
\begin{columns}[t]
  \column{.5\textwidth}
  \begin{tcolorbox}[height=45pt]
    \centering
    \vspace{5pt}
    $(\R, \id_{\R})$
  \end{tcolorbox}
  \column{.5\textwidth}
  \begin{tcolorbox}[height=45pt]
    \centering
    $(\R, \psi)$,\\
    $\psi(x) = x^3$
  \end{tcolorbox}
  \centering
\end{columns} \pause

  \begin{tcolorbox}[width=0.3\linewidth, colframe=red]
  \centering
  $ \id_{\R} \circ \psi^{-1}(y) = y^{\frac{1}{3}}$
  \end{tcolorbox}
\end{frame}

\begin{frame}
  \frametitle{Funciones Suaves}
  \begin{definition}[Función Suave]
    Sean $M$ una variedad suave  $n-$dimensional, $k$ un entero no negativo y $f: M \to \R^{k}$ una función cualquiera. Diremos que $f$ es una \textbf{función suave} si para cada punto $p \in M$ existe una carta suave $(U,\phi)$, cuyo dominio contiene a $p$ y tal que la composición de funciones $f \circ \phi^{-1}$ es suave en el conjunto abierto $\phi(U) \subseteq \R^{n}$.
  \end{definition} \pause

  Notemos que esta definición coincide con la definición usual de diferenciabilidad cuando $M = \R^n$.
\end{frame}

\begin{frame}
  \begin{figure}
    \begin{tikzpicture}[scale=0.85]
  \coordinate (a) at (0,0);
\path[draw,use Hobby shortcut,closed=true,thick]
(0,2.5) .. (2,2.5) .. (1,4.5) .. (.3,4.5) .. (-1,4) .. (-2,2.5);

\draw[dashed] (0.25,3.5) circle (0.6);
\draw node at (0,3.6) {$U$};
\draw node at (-1.5,4.5) {$M$};

\draw [thick, <->] (-5,0) -- (-1,0);
\draw [thick, <->] (-3,-2) -- (-3,2);
\draw [thick, <->] (1,0) -- (5,0);
\draw [thick, <->] (3,-2) -- (3,2);

\draw [dashed,thick] (-3,-0.25) ellipse  (1.5 and 1.2);


\draw[line width=1, ->] (1,3.75) arc (90:0:2.5);
\draw[line width=1, ->] (-0.75,3.5) arc (-90:0:-2);
\draw[line width=1, ->] (-1.5,-1.5) arc (60:120:-3.5);

\draw node at (-2.75,3) {$\phi$};
\draw node at (3.25,3) {$f$};
\draw node at (-2.25,-0.5) {$\phi(U)$};
\draw node at (-4.5,1) {$\R^n$};
\draw node at (4.5,1) {$\R^{k}$};
\draw node at (0.5,-1.25) {$f \circ \phi^{-1}$};
\end{tikzpicture}

    \caption{Representación de una función suave.}
  \end{figure}
\end{frame}

\begin{frame}
\frametitle{Mapas Suaves}
  De modo similar, podemos definir los mapas suaves explotando el hecho de que sabemos cuando una función que va de $\R^m$ a $\R^n$ es suave. \pause

\begin{definition}[Mapa Suave]
  Sean $M$ y $N$ variedades suaves, $F: M \to N$ un mapa cualquiera. Diremos que $F$ es un \textbf{mapa suave} si para cada $p \in M$ existen cartas $(U,\varphi)$ que contiene a $p$ y $(V,\psi)$ que contiene a $F(p)$ tal que $F(U)\subset V$ y la composición $\psi \circ F \circ \varphi^{-1}$ es suave de $\varphi(U)$ a $\psi(V)$.
\end{definition}
\end{frame}

\begin{frame}
  \begin{figure}
    \scalebox{.70}{\begin{tikzpicture}[scale=1]

%Variedad M
\path[draw,use Hobby shortcut,closed=true]
(-5,5.5) .. (-6,4) .. (-4,4) .. (-2,3.5) .. (-2.5,6);
\filldraw (-3,5) circle (0.05);
\node at (-3.25,5.25) {$p$};
\node at (-2.25,4.35) {$U$};
\draw[dashed] (-3,5) ellipse (0.6 and 0.8);
\node at (-1.25,5.75) {$M$};

%Variedad N
\path[draw,use Hobby shortcut,closed=true]
(5.5,6) .. (6,4) .. (4,3) .. (2,3.5) .. (2.5,5) .. (3.5,5.5);
\filldraw (3.25,4) circle (0.05);
\node at (2.75,3.6) {$F(p)$};
\node at (4.25,5) {$V$};
\draw[dashed](3.25,4) ellipse (1.2 and 0.9);
\node at (2,5.25) {$N$};

% Flecha M a N
\draw[thick,->] (-1,6) arc (-60:-130:-2.5);
\node at (0.25,6.75) {$F$};

% Flecha M a Rm
\draw[thick,->] (-3,4) arc (-10:20:-5);
\node at (-3.5,2.75) {$\phi$};

% Flecha N a Rn
\draw[thick,->] (3.75,3.25) arc (10:-20:4);
\node at (4.25,2) {$\psi$};

%Eje Izquierdo
\draw [<->] (-5,-1) -- (-1,-1);
\draw [<->] (-4,-1.5) -- (-4,2.5);
\node at (-5,2.5) {$\R^m$};
\node at (-1,1.25) {$\phi(U)$};
\draw[dashed] (-2.5, 0.25) circle (1);

%Eje Derecho
\draw [<->] (5,-1) -- (1,-1);
\draw [<->] (2,-1.5) -- (2,2.5);
\node at (1,2.5) {$\R^n$};
\node at (4.5,0) {$\psi(V)$};
\draw[dashed] (2.5, -0.25) circle (1.2);

% Flecha Rn a Rm
\draw[thick,->] (-2,-1.25) arc (60:120:-3.5);
\node at (0,-2) {$\psi \circ F \circ \phi^{-1}$};
\end{tikzpicture}
}
    \caption{Representación de un mapa suave.}
  \end{figure}
\end{frame}

\begin{frame}
\frametitle{Ejemplo de Funciones y Mapas Suaves}
  \begin{itemize}
    \item Las mapas constantes.
    \item El mapa identidad.
    \item El mapeo de inclusión
    \item Composición de funciones suaves.
    \item Las proyecciones.
    \item Toda función $f: \R^m \to \R^n$ que sea suave en el sentido de usual del cálculo será también un mapa suave en el sentido de variedades.
  \end{itemize}
\end{frame}
