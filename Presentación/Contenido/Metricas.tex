\begin{frame}
  Habiendo visto todo esto, ahora podemos hablar de como es que vamos a medir. \pause

  \begin{definition}[Métrica Riemanniana]
    Sea $M$ una variedad suave. Una \bf{Métrica Riemanniana} en $M$ es un campo suave 2-tensorial covariante simétrico que es definida positiva en cada punto.
  \end{definition} \pause

  ¿Qué quiere decir esto?
\end{frame}

\begin{frame}
  \begin{itemize}
    \item  Que $M$ sea una variedad suave significa que es un espacio topológico que localmente se ve como $\Bbb{R}^n$ y en el cual podemos darle sentido a las derivadas y definir un espacio tangente. \pause
    \item Que sea un campo suave nos está diciendo que toma funciones suaves en $M$ y las lleva a funciones suaves también en $M$, de modo que estás sean derivaciones. \pause
    \item Que sea un tensor de rango 2 covariante y simétrico quiere decir que toma como entradas dos elementos de los espacios tangentes y depende linealmente de cada uno de ellos y permanece invariante al cambiar su orden. \pause
    \item Por ultimo, que sea positiva definida quiere decir que si la función toma en sus dos argumentos al mismo elemento, entonces el resultado es mayor o igual a cero.
  \end{itemize}
\end{frame}

\begin{frame}
  \begin{definition}[Variedad Riemanniana]
    Una \bf{Variedad Riemanniana} es un par $(M,g)$, donde $M$ es una variedad suave y $g$ es una métrica en $M$.
  \end{definition}

  Si $g$ es una métrica Riemanniana en $M$, entonces para cada punto $p \in M$, el 2-tensor $g_p$ es un producto interno en $T_p M$, por lo que es usual escribir $\langle v,w \rangle_g$ para denotar al número real $g_p (v,w)$
\end{frame}

\begin{frame}
  \begin{example}[Métrica Euclidiana]
  El ejemplo más simple de una métrica de Riemann es es la \bf{Métrica Euclidiana} $\bar{g}$ en $\Bbb{R}^n$, dada en la coordenadas estándar por

  \[
    \bar{g} = \delta_{ij}dx^{i}dx^j
  \]

  donde $\delta_{ij}$ es la delta de Kronecker. Es común abreviar el producto simétrico de un tensor $\alpha$ consigo mismo como $\alpha^2$, por lo que la métrica Euclidiana puede ser escrita como:
  \[
    \bar{g} = (dx^1)^2 + \hdots + (dx^n)^2
  \]

  Aplicado a vectores $v,w \in T_p\Bbb{R}^n$, esto nos da:
  \[
    \bar{g}_p (v,w) = \delta_{ij}v^i w^j = \sum_{i=1}^n v^i w^i = v \cdot w
  \]
  \end{example}
\end{frame}

\begin{frame}
  \begin{theorem}[Existencia de Métricas Riemannianas]    
    Cada variedad suave admite una métrica Riemanniana.
  \end{theorem} \pause

  Es importante notar que hay muchas maneras de construir una métrica $g$ para una variedad dada, y que, si tenemos métricas diferentes en la misma variedad, estás pueden tener propiedades geométricas completamente diferentes.
\end{frame}

\begin{frame}
  Algunas de las construcciones que podemos definir utilizando en una variedad Riemanniana son las siguientes:\pause
 \begin{itemize}
   \item La \bf{Longitud} o \bf{Norma} de un vector tangente $v \in T_p M$ se define como:
     \[
       |v|_g = \langle v,v \rangle_g^{\frac{1}{2}} = g_p (v,v)^{\frac{1}{2}}
     \] \pause
   \item El \bf{Ángulo} entre dos vectores tangentes no nulos $v,w \in T_p M$ está dado como el $\vartheta \in [0,\pi]$ único que satisface:
     \[
       \cos \varphi = \frac{\langle v,w \rangle_g}{|v|_g |w|_g}
     \] \pause
    \item Dados dos vectores tangentes $v,w \in T_pM$, decimos que estos son \bf{Ortogonales} si $\langle v,w\rangle_g = 0$.
 \end{itemize} 
\end{frame}
